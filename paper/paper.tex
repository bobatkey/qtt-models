\documentclass[acmsmall,review]{acmart}
\settopmatter{printfolios=true,printccs=false,printacmref=false}

\usepackage{cmll}
\usepackage{mathpartir}

\newcommand{\TapeType}{\mathrm{Tape}}

\newcommand{\tmRec}{\mathrm{rec}}
\newcommand{\tyNat}{\mathrm{Nat}}
\newcommand{\conZero}{\mathsf{zero}}
\newcommand{\conSucc}{\mathsf{succ}}
\newcommand{\conRefl}{\mathsf{refl}}

\newcommand{\Let}{\mathrm{let}}
\newcommand{\LetPair}{\mathrm{letpair}}
\newcommand{\In}{\mathrm{in}}
\newcommand{\If}{\mathrm{if}}
\newcommand{\Then}{\mathrm{then}}
\newcommand{\Else}{\mathrm{else}}
\newcommand{\dupNat}{\mathrm{dupNat}}
\newcommand{\Match}{\mathrm{match}}

% QTT stuff
\newcommand{\istype}{\mathrm{type}}
\newcommand{\isctxt}{\mathrm{ctxt}}

\newcommand{\BoolTy}{\mathrm{Bool}}
\newcommand{\cTrue}{\mathsf{true}}
\newcommand{\cFalse}{\mathsf{false}}
\newcommand{\ListTy}{\mathrm{List}}
\newcommand{\cNil}{\mathsf{nil}}
\newcommand{\cCons}{\mathsf{cons}}

\newcommand{\Rtype}{\mathbf{R}}
\newcommand{\rIntro}{\mathbf{R}}
\newcommand{\rElim}{\mathbf{R}^{-1}}

\newcommand{\natinf}{\mathbb{N}_{-\infty}}

\newcommand{\Ty}{\mathrm{Ty}}
\newcommand{\RTm}{\mathrm{RTm}}
\newcommand{\Tm}{\mathrm{Tm}}

\newcommand{\Set}{\mathrm{Set}}

\newcommand{\cat}[1]{\mathcal{#1}}
\newcommand{\op}{\mathsf{op}}

\newcommand{\LinPreorder}{\mathrm{LinPreorder}}

\newcommand{\bob}[1]{\textcolor{purple}{FIXME: #1}}

\renewcommand{\sectionautorefname}{Section}
\renewcommand{\subsectionautorefname}{Section} % PS
\renewcommand{\subsubsectionautorefname}{Section}


%% Rights management information.  This information is sent to you
%% when you complete the rights form.  These commands have SAMPLE
%% values in them; it is your responsibility as an author to replace
%% the commands and values with those provided to you when you
%% complete the rights form.
\setcopyright{acmcopyright}
\copyrightyear{2020}
\acmYear{2020}
\acmDOI{10.1145/1122445.1122456}


%%
%% These commands are for a JOURNAL article.
\acmJournal{PACMPL}
\acmVolume{1}
\acmNumber{CONF}
\acmArticle{1}
\acmMonth{8}

%%
%% Submission ID.
%% Use this when submitting an article to a sponsored event. You'll
%% receive a unique submission ID from the organizers
%% of the event, and this ID should be used as the parameter to this command.
%%\acmSubmissionID{123-A56-BU3}

%%
%% The majority of ACM publications use numbered citations and
%% references.  The command \citestyle{authoryear} switches to the
%% "author year" style.
%%
%% If you are preparing content for an event
%% sponsored by ACM SIGGRAPH, you must use the "author year" style of
%% citations and references.
%% Uncommenting
%% the next command will enable that style.
%%\citestyle{acmauthoryear}

%%
%% end of the preamble, start of the body of the document source.
\begin{document}

%%
%% The "title" command has an optional parameter,
%% allowing the author to define a "short title" to be used in page headers.
\title{Polynomial Time and Full Dependent Types}

%%
%% The "author" command and its associated commands are used to define
%% the authors and their affiliations.
%% Of note is the shared affiliation of the first two authors, and the
%% "authornote" and "authornotemark" commands
%% used to denote shared contribution to the research.
\author{Robert Atkey}
\email{robert.atkey@strath.ac.uk}
\orcid{1234-5678-9012}
\affiliation{%
  \institution{University of Strathclyde}
  \streetaddress{26 Richmond Street}
  \city{Glasgow}
  \country{UK}
  \postcode{G1 1XH}
}


%%
%% By default, the full list of authors will be used in the page
%% headers. Often, this list is too long, and will overlap
%% other information printed in the page headers. This command allows
%% the author to define a more concise list
%% of authors' names for this purpose.
%\renewcommand{\shortauthors}{Trovato and Tobin, et al.}

%%
%% The abstract is a short summary of the work to be presented in the
%% article.
\begin{abstract}
  We combine dependent types for reasoning about programs with linear
  type systems for implicit polynomial time.
\end{abstract}

%%
%% The code below is generated by the tool at http://dl.acm.org/ccs.cfm.
%% Please copy and paste the code instead of the example below.
%%
\begin{CCSXML}
<ccs2012>
 <concept>
  <concept_id>10010520.10010553.10010562</concept_id>
  <concept_desc>Computer systems organization~Embedded systems</concept_desc>
  <concept_significance>500</concept_significance>
 </concept>
 <concept>
  <concept_id>10010520.10010575.10010755</concept_id>
  <concept_desc>Computer systems organization~Redundancy</concept_desc>
  <concept_significance>300</concept_significance>
 </concept>
 <concept>
  <concept_id>10010520.10010553.10010554</concept_id>
  <concept_desc>Computer systems organization~Robotics</concept_desc>
  <concept_significance>100</concept_significance>
 </concept>
 <concept>
  <concept_id>10003033.10003083.10003095</concept_id>
  <concept_desc>Networks~Network reliability</concept_desc>
  <concept_significance>100</concept_significance>
 </concept>
</ccs2012>
\end{CCSXML}

\ccsdesc[500]{Computer systems organization~Embedded systems}
\ccsdesc[300]{Computer systems organization~Redundancy}
\ccsdesc{Computer systems organization~Robotics}
\ccsdesc[100]{Networks~Network reliability}

%%
%% Keywords. The author(s) should pick words that accurately describe
%% the work being presented. Separate the keywords with commas.
\keywords{type theory, implicit computational complexity, linear types}


%%
%% This command processes the author and affiliation and title
%% information and builds the first part of the formatted document.
\maketitle

%%%%%%%%%%%%%%%%%%%%%%%%%%%%%%%%%%%%%%%%%%%%%%%%%%%%%%%%%%%%%%%%%%%%%%
%%%%%%%%%%%%%%%%%%%%%%%%%%%%%%%%%%%%%%%%%%%%%%%%%%%%%%%%%%%%%%%%%%%%%%
%%%%%%%%%%%%%%%%%%%%%%%%%%%%%%%%%%%%%%%%%%%%%%%%%%%%%%%%%%%%%%%%%%%%%%
%%%%%%%%%%%%%%%%%%%%%%%%%%%%%%%%%%%%%%%%%%%%%%%%%%%%%%%%%%%%%%%%%%%%%%
\section{Introduction}
\label{sec:introduction}

{\bf New story:}
\begin{enumerate}
\item Introduction: The goal of this paper is to unify implicit
  computational complexity with full dependent types. ICC enables us
  to ensure that \emph{programs} are within a certain complexity
  class, while dependent types allow us to state and prove that these
  programs solve the problems that we intend them to, and to construct
  reductions between problems. We can marry complexity-controlled
  implementations with problems, paving the way for a \emph{synthetic
    computational complexity theory}.
\item Section 1: Linear types and Implicit Polynomial Time.
  \begin{enumerate}
  \item Total typed functional programming, as we see in type theory,
    is restricted to the use of higher order functions and iteration
    over inductive data types.
  \item Polynomial time is the smallest class closed under nested
    iteration over the input -- i.e. we consider iteration over the
    input as feasible, and compositions thereof. The key thing is to
    avoid iteration over data constructed from the input.
  \item Key is to control iteration. So we split inductive datatypes
    into iterable and non-iterable versions. We can
  \item One way to capture polytime is to ensure that all iteration is
    only done over the input. So we can duplicate iterable things, but
    we can't construct new iterable things (at least not from an
    iteration). We also can't duplicate higher order things, because
    that effectively acts as a way to duplicate iterations (half
    instantiated iterators). Compare to Jones' CONS-FREE programming.
  \item Another approach is to allow ``borrowing'' of iterability by
    means of diamonds. Iterability becomes a first-class object, so we
    can use it to construct new data structures from the input. This
    allows the construction of new iterable data structures. LFPL.
  \end{enumerate}
\item Section 2: Extending to Dependent Types
  \begin{enumerate}
  \item Combining linear types and dependent types is subtle, because
    there is a fundamental incompatibility in the syntax of the two
    systems.
  \item There have been several approaches, though they are all
    somewhat similar (list them all here). We will use QTT, because it
    offers a firm semantic foundation as well as a relatively
    straightforward syntax.
  \item Recap the syntax of QTT, with extensional equality. Key point
    is that there is a normal dependently typed language with a linear
    sub-language. We can have special constructs in the linear
    sub-language that ``erase'' to normal type theory -- products are
    the prime example.
  \item Non iterable datatypes -- have normal eliminators in the DTT
    fragment, but
  \end{enumerate}
\item Section 3: Programming and Proving with Poly-time
  \begin{enumerate}
  \item Internalised completeness proofs.
  \item Construct the type of problems that are solvable in polynomial
    time.
  \item Construct reductions between problems.
  \item NP. Doesn't make sense to talk about this without dependent
    types, because we can't say when we've solved the problem being
    stated.
  \item PP (Probabilistic), using a monad. Again, need to be able to
    state connection with the problem to know that we are solving it.
  \end{enumerate}
\item Section 4: Soundness via Realisability
  \begin{enumerate}
  \item The underlying CBV language and its
  \end{enumerate}
\item Section 5: Extensions
  \begin{enumerate}
  \item Explicit Complexity via Diamond Counting
  \item Poly-time univalence
  \item Towards P- and NP-completeness proofs. Issue here is the
    extensionality. Towards a synthetic (poly-time based)
    computational complexity theory.
  \end{enumerate}
\item Section 6: Related Work
  \begin{enumerate}
  \item ICC stuff
    \begin{enumerate}
    \item LFPL
    \item CONS-free
    \item Soft Affine Logic; which has the same underlying idea as
      CONS-free (is this a novel observation?)
    \item Other systems based on information flow.
    \end{enumerate}
  \item Explicit complexity: lambda-amor, and things by Deepak Garg
    and Marco Gaboardi.
  \item Recent CMU stuff on calf
  \end{enumerate}

\end{enumerate}

{\bf Old story:}
\begin{enumerate}
\item There are systems for implicit computational complexity based on
  linear logic (e.g. LFPL, SAL, LLL, BLL) that seek to capture
  complexity classes (most often polynomial time) implicitly
  (i.e. without directly talking about time bounds). This work has
  also led to followup work on explicitly resourced calculi such as
  RAML, which seek to use static analysis to determine complexity
  bounds on programs.
\item In dependent type theory, we can reason about the extensional
  behaviour of programs (within the limits of intensional type
  theory), but not about intenstional properties. The only way to
  formalise something like ``for all polynomial time functions
  $\mathbb{N} \to \mathbb{N}$'' is to develop a deep emebdding of
  polynomial time functions and to work with that, which is tedious.
\item The goal of this paper is to unify dependent types for reasoning
  about programs with implicit computational complexity type
  systems. In the first instance, we do not seek to use dependent
  types to state precise bounds on programs' time consumption, but we
  intend to use dependent types to reason about the extensional
  behaviour of programs that we know to be polynomial time. At the end
  of the paper, we show how to adapt our framework to explicit
  resource accounting, in the style of RAML.
\item The plan of the paper is as follows:
  \begin{enumerate}
  \item We first introduction the ideas behind the use of linear logic
    for capturing polynomial time, by studying two systems Lafont's
    SAL, and Hofmann's LFPL. In Hofmann and Dal Lago's presentation,
    these are two variants of Second order Multiplicative Linear
    Logic, with restricted $\oc$ modalities for duplication. We also
    recall the realisability semantics of Dal Lago and Hofmann (FIXME:
    move this later?)
  \item The most modern presentations of these systems use second
    order quantification to represent data types. This is incompatible
    with the form of dependent types we use here, so we seek a direct
    representation of natural numbers that captures
  \item We then recall QTT, a extension of Martin-Löf type theory with
    linear types, which can accomodate intensional restrictions on
    computability as well as full reasoning about programs. Since
    allowing an unrestricted $\oc$ modality would violate our
    polynomial time bounds, we instead choose the natural number
    semiring for resource accounting.
  \item We then transplant the simply typed rules into QTT, showing
    how to carefully distinguish between data that is available to the
    program, and data that is only available for type formation.
  \item We finish by showing that we can also do explicit resource
    tracking.
  \end{enumerate}
\end{enumerate}

Related work: only applications of Linear TT so far are erasure (Abel
et al.) and quantum stuff (Selinger et al.), and (sort of) dependency
analysis. This is another one.

%%%%%%%%%%%%%%%%%%%%%%%%%%%%%%%%%%%%%%%%%%%%%%%%%%%%%%%%%%%%%%%%%%%%%%
%%%%%%%%%%%%%%%%%%%%%%%%%%%%%%%%%%%%%%%%%%%%%%%%%%%%%%%%%%%%%%%%%%%%%%
%%%%%%%%%%%%%%%%%%%%%%%%%%%%%%%%%%%%%%%%%%%%%%%%%%%%%%%%%%%%%%%%%%%%%%
%%%%%%%%%%%%%%%%%%%%%%%%%%%%%%%%%%%%%%%%%%%%%%%%%%%%%%%%%%%%%%%%%%%%%%
\section{Linear $\lambda$-Calculus and Polytime}
\label{sec:linear-intro}

Not long after Girard introduced Linear Logic \cite{girard87}, it was
observed that its resource sensitivity could be turned to describing
computational complexity classes by purely logical means. Typically, a
logical system is described for which the process of reducing a proof
to a normal form (often by cut elimination) is guaranteed to always be
accomplished within a certain complexity bound. Moreover, the system
is usually proven to be complete for the relevant complexity class, by
constructing a simulation of some known representation. Such systems
that characterise polytime include Bounded Linear Logic (BLL), which
uses explicit polynomials in the formulas \cite{bll} and Soft Affine
Logic (SAL) \cite{lafont}, which does not explicitly represent
time information in formulas, but uses a restricted form of Linear
Logic's $\oc$ modality instead. Light Affine Logic (LAL) \cite{lal} is
another ``counting-free'' system for polytime.

Viewing logical systems though the lens of the Curry-Howard
correspondence, the idea arises that one could define functional
programming languages that characterise complexity classes such as
polytime. SAL has been transformed into a programming language for
polytime by \cite{gaboardi}. Hofmann proposed a new programming
language, Linear Functional Programming Language (LFPL), that uses a
novel ``payment'' system to track iteration. In this section, we
review the use of linear types to capture polytime by presenting two
systems, one based on ideas from SALL and the second more explicitly
based on LFPL. With a view to adapting these systems to dependent
types in \autoref{sec:qtt}, we take an approach slightly different to
much of the polytime linear logic literature. We emphasise the use of
explicit datatypes and eliminators, rather than using impredicative
encodings via universal types. Thus, we are closer to Hofmann's
original LFPL (though not later presentations of it
\cite{dallago-hofmann}) than BLL, SALL, or LAL.

For this section, the linear $\lambda$-calculus we will use will have
linear functions and $\otimes$-products:
\begin{mathpar}
  \inferrule*
  { }
  {x : A \vdash x : A}

  \inferrule*
  {\Gamma, x : A \vdash M : B}
  {\Gamma \vdash \lambda x.M : A \multimap B}

  \inferrule*
  {\Gamma_1 \vdash M : A \multimap B \\
    \Gamma_2 \vdash N : A}
  {\Gamma_1, \Gamma_2 \vdash M\,N : B}

  \inferrule*
  {\Gamma_1 \vdash M : A \\ \Gamma_2 \vdash N : B}
  {\Gamma_1, \Gamma_2 \vdash (M,N) : A \otimes B}

  \inferrule*
  {\Gamma_1 \vdash M : A \otimes B \\ \Gamma_2, x : A, y : B \vdash N : C}
  {\Gamma_1, \Gamma_2 \vdash \Let\,(x,y)=M\,\In\,N : C}
\end{mathpar}
These rules are standard, so we do not describe them further except to
note how linear typing uses presence or absences in a context to
control resource usage -- if a variable is in the context it must be
used exactly once. The fact that this discipline interferes with
dependent types is one of the reasons we turn to QTT when we wish to
add dependent types in \autoref{sec:qtt}.

\subsection{No Recursion, Only Case Analysis}
\label{sec:linear-case-analysis}

It is not too difficult to see that reduction of linear
$\lambda$-terms always takes a number of steps linearly proprotional
to the size of the term. This is because every $\beta$-redex
substitutes into exactly one variable, reducing the size of the term
by one each time.

We can increase the expressivity, but not the computational
complexity, of the system by adding datatypes that do not allow
iteration. These can be used for representation, but not for driving
computation. We include the rules here, to show how linearity must be
preserved in these rules, and to foreshadow their dependently typed
counterparts in \autoref{sec:qtt-noniter}. The first type is the
booleans, which are anyway non-recursive and so wouldn't allow
iteration anyway:
\begin{mathpar}
  \inferrule*
  { }
  {\vdash \cTrue, \cFalse : \BoolTy}

  \inferrule*
  {\Gamma_1 \vdash M : \BoolTy \\ \Gamma_2 \vdash N_1 : A \\ \Gamma_2 \vdash N_2 : A}
  {\Gamma_1, \Gamma_2 \vdash \If\,M\,\Then\,N_1\,\Else\,N_2 : A}
\end{mathpar}
The if-then-else rule is careful to ensure that the resources used by
the eliminated $\BoolTy$ and the resources used by the chosen branch
are accounted for separately. The two branches must have the same
resource usage.

Construction and case analysis of lists are given by the following
rules:
\begin{mathpar}
  \inferrule*
  { }
  {\vdash \cNil : \ListTy(A)}

  \inferrule*
  {\Gamma_1 \vdash M : A \\ \Gamma_2 \vdash N : \ListTy(A)}
  {\Gamma_1, \Gamma_2 \vdash \cCons(M,N) : \ListTy(A)}

  \inferrule*
  {\Gamma_1 \vdash M : \ListTy(A) \\
    \Gamma_2 \vdash N_1 : B \\
    \Gamma_2, h : A, t : \ListTy(A) \vdash N_2 : B}
  {\Gamma_1, \Gamma_2 \vdash \Match\,M\,\{\cNil \mapsto N_1; \cCons(h,t) \mapsto N_2\} : B}
\end{mathpar}
Hence, we can construct lists arbitrarily, but we can only do case
analysis on them. If we wish to explore a list to a arbitrary depth it
must be driven by an iterable type of the form that we will see in the
next two sections.

With booleans and lists, we can construct several other useful
types. For example, to simulate Turing machines, one can construct a
$\mathit{Tape}$ type as a Zipper \cite{huet}
$\ListTy(\BoolTy) \otimes \BoolTy \otimes \ListTy(\BoolTy)$,
representing a position on the tape with the items before, under, and
after the head.

\subsection{The Cons-free system}
\label{sec:cons-free-intro}

Polynomial time is usually seen as a proxy for ``feasible''
computation. On the face of it, there does not seem to be any
particular reason why polynomials have anything to do with
feasibility. However, one can arrive at the definition of polynomial
time in three steps, by assuming that (i) iterating over the whole
input is feasible; (ii) if two computations are feasible, then so is
their composition; and (iii) performing a feasible computation for
every element of the input is also feasible. It is the last point that
allows complexities of arbitrary polynomial degree to be constructed
(we will see this in action in the completeness construction below and
soundness proofs in \autoref{sec:iterator-soundness}).

Following these ideas, let us assume that the input is a natural
number, so we assume that there is some type of natural numbers
$\tyNat$. For point (i), we must be able to iterate over these natural
numbers, so we use a linear iterator defined by this typing rule:
\begin{displaymath}
  \inferrule*
  {\vdash M_z : A \\ x : A \vdash M_s : A \\ \Gamma \vdash N : \tyNat}
  {\Gamma \vdash \tmRec\,N\,\{\conZero \mapsto M_z; \conSucc(x) \mapsto M_s\} : A}
\end{displaymath}
Note that in the zero, $M_z$, and sucessor, $M_s$, cases, the context
is empty to ensure that these cases may be invoked as many times as
required. Point (ii) above is automatically satisfied by being in a
typed $\lambda$-calculus, where it is difficult to stop functions from
being composable. For point (iii), the iterator as given does not
allow us to nest iterations. Once the natural number input $n$ has
been used for an iteration, the linear typing discipine prevents us
from using it again (note the two separate contexts $\Gamma_1$,
$\Gamma_2$ in the rule for application). In order to allow nested
iterations, we add an operator to duplicate numbers:
\begin{displaymath}
  \inferrule*
  {\Gamma \vdash M : \tyNat}
  {\Gamma \vdash \dupNat\,M : \tyNat \otimes \tyNat}
\end{displaymath}
Somewhat surprisingly, this system is now sound and complete for
polynomial time. Crucially, this depends on the two things we have
\emph{not} allowed. First, we have disallowed the construction of new
natural numbers via the $\conZero$ and $\conSucc$
constructors\footnote{Actually, $\conZero$ would be acceptable, as
  well as any constant natural number. It is only unrestricted use of
  $\conSucc$ that is dangerous.}. If we were to permit this, then we
could use iteration over the input to construct addition,
multiplication (by repeated addition), and then exponentials (by
repeated multiplication). We therefore refer to this system as the
\emph{Cons-Free} system.  The second prohibited feature is the ability
to duplicate values of function type, even though we have allowed
duplication of iterable naturals. If we were to allow this, then we
would be able to sneak in a form of constructors for natural numbers
by encoding them as eliminators that duplicate a function for every
$\conSucc$ step.

We will see in \autoref{sec:consfree-sound} that this system is sound
for polytime by a realisability argument. Completeness can be seen
more directly by constructing a function that iterates a function for
a statically known polynomial number of times in the size of the
input. Assume that we have a known polynomial
$p(n) = c_dn^d + \dots + c_0$ of degree $d$ with natural number
coefficients and some single step function
$f : \mathit{St} \multimap \mathit{St}$ over a state type
$\mathit{St}$ that runs to completion for input of size $n$ in $p(n)$
steps. Then, using the iterator above we can iterate $f$ over a
$\tyNat$ representing the size of the input:
\begin{displaymath}
  \begin{array}{l}
    I_1 : \tyNat \multimap \mathit{St} \multimap \mathit{St} \\
    I_1 = \lambda n. \lambda s.\tmRec\,n\,\{\conZero \mapsto s; \conSucc(s) \mapsto f\,s \}
  \end{array}
\end{displaymath}
To achieve higher degrees, we can use $\dupNat$ to nest iterations:
\begin{displaymath}
  \begin{array}{l}
    I_{k+1} : \tyNat \multimap \mathit{St} \multimap \mathit{St} \\
    I_{k+1} = \lambda n. \lambda s.
    \begin{array}[t]{@{}l}
      \Let\,(n,n') = \dupNat\,n\,\In\\
      \tmRec\,n\,\{\conZero \mapsto s; \conSucc(s) \mapsto I_k\,n'\,s \}
    \end{array}
  \end{array}
\end{displaymath}
By further use of $\dupNat$ and composition to handle addition of
polynomials, the function $f$ can now be iterated $p(n)$ many times,
where $n$ is the input $\tyNat$. Thus, the \emph{Cons-free} system can
represent all polytime computations.

\subsection{Diamond Trading with LFPL}
\label{sec:lfpl-intro}

The Cons-free system is sound and complete for polytime, but is quite
awkward from the point of view of functional programming. It allows us
to iterate over natural numbers that come from the input but does not
allow us to build further values to do iteration on. For example, if
our input is a list, then we cannot transform it into a binary search
tree and then flatten it, we must always refer back to the original
natural number input. Even dividing the input into two halves to be
treated separately is difficult.

A more flexible system was proposed by \cite{hofmann99}. The
\emph{Linear Functional Programming Language} (LFPL) takes a different
approach to the \emph{Cons-free} system. Instead of completely
prohibiting construction of new data, constructions is allowed if it
is paid for by values of new type $\Diamond$ (``diamonds''):
\begin{mathpar}
  \inferrule*
  {\Gamma \vdash M : \Diamond}
  {\Gamma \vdash \conZero(M) : \tyNat}

  \inferrule*
  {\Gamma_1 \vdash M : \Diamond \\ \Gamma_2 \vdash N : \tyNat}
  {\Gamma_1, \Gamma_2 \vdash \conSucc(M, N) : \tyNat}
\end{mathpar}
Thus, to construct a $\conZero$, we must have a $\Diamond$ to pay for
it, and likewise, to construct a $\conSucc$ we must have a diamond. We
can think of $\Diamond$s as an unit of iterable data. Iterability is
``saved up'' in data during construction, and released during
iteration. Diamonds cannot be created by a program itself, for the
same reason that constructors were prohibited in the \emph{Cons-free}
system, but they are released from iterable data during iteration. The
LFPL natural number iterator has the following typing rule:
\begin{displaymath}
  \inferrule*
  {d : \Diamond \vdash M_z : A \\ d : \Diamond, x : A \vdash M_s : A \\ \Gamma \vdash N : \tyNat}
  {\Gamma \vdash \tmRec\,N\,\{\conZero(d) \mapsto M_z; \conSucc(d,x) \mapsto M_s \} : A}
\end{displaymath}
The difference with the \emph{Cons-free} iterator above is that the
$\conZero$ and $\conSucc$ cases now both have an addition binding of
type $\Diamond$. This allows some form nesting of iterations: during
an iteration over the input, the program can accumulate $\Diamond$s to
use for iteration over substructures that are smaller than the current
point in the iteration. A construction, due to Aelhig and
Schwictenberg \cite{syntactic-lfpl}, illustrates how this leads to all
polytime computations. As above, we assume a polynomial $p(n)$ and a
step function $f : \mathit{St} \multimap \mathit{St}$ that needs to be
iterated $p(n)$ times. We construct a linear iterator:
\begin{displaymath}
  \begin{array}{l}
    I_1 : (\tyNat \otimes \mathit{St}) \multimap (\tyNat \otimes \mathit{St}) \\
    I_1 = \lambda (n, s).\,\tmRec\,n\,\{
    \begin{array}[t]{@{}lcl}
      \conZero(d)&\mapsto&(\conZero(d), s);\\
      \conSucc(d,(n,s)) &\mapsto& (\conSucc(d,n),f\,s) \}
    \end{array}
  \end{array}
\end{displaymath}
Note that this iterator returns the natural number input as well as
the new state. LFPL does not allow duplication of iterable inputs, so
we must always reconstruct it if we want to do further
iteration. Addition of polynomials is accomplished by composition of
iterators. To raise the degree, we again use a nesting iterator:
\begin{displaymath}
  \begin{array}{l}
    I_{k+1} : (\tyNat \otimes \mathit{St}) \multimap (\tyNat \otimes \mathit{St}) \\
    I_{k+1} = \lambda (n,s).\,\tmRec\,n\,\{
    \begin{array}[t]{@{}lcl}
      \conZero(d)&\mapsto&(\conZero(d),s); \\
      \conSucc(d,(n,s)) &\mapsto& \Let\,(n,s)=I_k\,(n,s)\,\In\,(\conSucc(d,n), f\,s) \}
    \end{array}
  \end{array}
\end{displaymath}
Unlike in the \emph{Cons-free} system, this iterator does not raise
the degree of the nested iterator directly. Rather the iterator $I_k$
performs $\binom{n}{k}$ iterations. As observed by Aelhig and
Schwictenberg, this is sufficient because the binomials form a basis
for the vector space of all polynomials.

Despite this slightly more involved completeness construction, the
advantage of LFPL is that it is now easy to have arbitrary iterable
datatypes and to transform between them. We need only take the
introduction and elimination rules for any inductive datatype and add
$\Diamond$ premises to the introduction rules and $\Diamond$ bindings
to the eliminators.


% \subsection{Explicit Resource Tracking}

% \begin{enumerate}
% \item We still require diamonds to build zero and successor, but the
%   iterator does not give them back.
% \item In the simply typed case, we can annotate types with the static
%   amount of potential they have per constructor.
% \end{enumerate}

%%%%%%%%%%%%%%%%%%%%%%%%%%%%%%%%%%%%%%%%%%%%%%%%%%%%%%%%%%%%%%%%%%%%%%
%%%%%%%%%%%%%%%%%%%%%%%%%%%%%%%%%%%%%%%%%%%%%%%%%%%%%%%%%%%%%%%%%%%%%%
%%%%%%%%%%%%%%%%%%%%%%%%%%%%%%%%%%%%%%%%%%%%%%%%%%%%%%%%%%%%%%%%%%%%%%
%%%%%%%%%%%%%%%%%%%%%%%%%%%%%%%%%%%%%%%%%%%%%%%%%%%%%%%%%%%%%%%%%%%%%%
\section{Polytime Quantitative Type Theory}
\label{sec:qtt}

We have now seen the \emph{Cons-free} and \emph{LFPL} systems for
capturing polytime by means of linear typing and restricted
iteration. We now look to extend these systems to include dependent
types.

To integrate linear typing for controlling time complexity with
dependent types we use \emph{Quantitative Type Theory} (QTT)
\cite{atkey18qtt}. We now review the definition of QTT and describe
how we have adapted it to the polytime systems we saw in the previous
section.

\subsection{Quantitative Type Theory}

Integrating linear and dependent types is not straightforward due to
the conflict between the linear typing discipline regarding presence
of a variable as only bestowing the right to use it once, and the
dependent typing regime that uses variables both in types (for
specification purposes) and in terms (for computational purposes),
syntactically yield multiple uses of the same variable.

QTT is a system that resolves this conflict by recording usage of
variables with annotations from a semiring. It sits in the general
area of systems that use semiring annotatons to measure resource usage
\cite{BLAH}. The key feature of QTT, originally due to
\cite{mcbride16}, is that usage of variables in types counts for
$0$-usage in terms of the semiring used. This allows us to use normal
type theory as a specification language, while also enjoying the
benefits of linear typing for programs. The term typing judgement of
QTT has the following form:
\begin{displaymath}
  x_1 \stackrel{\rho_1}: S_1, \dots, x_n \stackrel{\rho_n}: S_n \vdash M \stackrel\sigma: T
\end{displaymath}
where the annotations $\rho_i$ are all from the semiring being
used. The annotation $\sigma$ is either $0$ or $1$, indicating whether
we are in the erased ($\sigma = 0$) fragment, where all the normal
rules of type theory apply, or the in the non-erased (``present'',
$\sigma = 1$) fragment, where a restricted typing discipline applies.
As we shall see below in the cases of $\Sigma$-types, iterable types,
and LFPL's $\Diamond$ type, the separation of QTT into these two
fragments allows an expressive combination of reasoning using full
type theory with the benefits of linear typing.

In the remainder of this sub-section, we describe the core of QTT. As
well as the term typing judgement given above, QTT also has judgements
for well-formed contexts ($\Gamma~\isctxt$) and types
($\Gamma \vdash T~\istype$), and definition equality of types and
terms ($\Gamma \vdash S \equiv T~\istype$ and
$\Gamma \vdash M \equiv N \stackrel\sigma: S$). It is an invariant of
the system that types are always well-formed in a context with all
annotations $0$, i.e., $\Gamma \vdash S~\istype$ implies
$0\Gamma = \Gamma$. An important admissible rule of the system, along
with weakening and substitution, is that of $0$-ing:
\begin{displaymath}
  \inferrule*
  {\Gamma \vdash M \stackrel1: S}
  {0\Gamma \vdash M \stackrel0: S}
\end{displaymath}
This rule allows us to take any term $M$ in the $\sigma = 1$ fragment
and treat it as if it were in the $\sigma = 0$ fragment, and hence use
it for specification purposes in types. As we add novel rules to QTT
in the following sections, we will be careful to maintain the
admissibility of this rule.

\subsubsection{Natural-number usages}

We use an instantiation of QTT with the natural number semiring, with
the usual semiring structure of addition and multiplication. In a mild
generalisation of the original presentation of QTT, we also allow
sub-usaging via the \emph{reverse} ordering on the naturals. That is,
if a variable is marked as usage $n$ and $m \geq n$, then we can also
regard it as usage $m$. This makes the system more like affine linear
logic, since $m \leq 0$ for all $m$, matching the system in
\autoref{sec:linear-intro}. We do \emph{not} have an unrestricted
usage $\omega$, since this would allow the possibility of unrestricted
duplication, and hence violate our polytime soundness properties.

\subsubsection{Contexts, variables, and conversion}

As we saw above, contexts in QTT are comprised of variable
$\stackrel\rho:$ type triples, where $\rho$ is a natural number
indicating how many times the variable $x$ is available for use in a
$\sigma = 1$ term. There are two operations on raw contexts: scaling
$\pi\Gamma$, which multiplies each $\rho$ in $\Gamma$ by $\pi$, and
addition $\Gamma_1 + \Gamma_2$, which adds two contexts' usage
annotations assuming that the lengths and types are equal.  The basic
usage-annotation discipline of QTT is demonstrated by the context
formation and variable rules:
\begin{mathpar}
  \inferrule*
  { }
  {\epsilon~\isctxt}

  \inferrule*
  {\Gamma~\isctxt \\ 0\Gamma \vdash S~\istype}
  {\Gamma, x \stackrel\rho: S~\isctxt}

  \inferrule*
  {0\Gamma, x \stackrel\sigma: S, 0\Gamma'~\isctxt}
  {0\Gamma, x \stackrel\sigma: S, 0\Gamma' \vdash x \stackrel\sigma: S}
\end{mathpar}
As with most dependent type theories, contexts are built inductively
from the empty context $\epsilon$ and extension of a context by a
variable with a type that is well-formed in the preceding
context. Usage annotations $\rho$ on variables are arbitrary, and
types are always judged in a $0$-annotated context. The variable rule
marks unused variables as usage $0$ and the selected variable with
usage $\sigma$.

% \begin{mathpar}

%   \inferrule* [right=Tm-Conv]
%   {\Gamma \vdash M \stackrel\sigma: S \\ 0\Gamma \vdash S \equiv T~\istype}
%   {\Gamma \vdash M \stackrel\sigma: T}
% \end{mathpar}

\subsubsection{$\Pi$- and $\Sigma$-types}

QTT's $\Pi$-types have the form $(x \stackrel\rho: S) \to T$,
indicating functions that, in the $\sigma = 1$ fragment, use their
arguments $\rho$-many times. The formation, introduction and
elimination rules are similar to the standard ones, except for the
addition of usage annotations:
\begin{mathpar}
  \inferrule*
  {0\Gamma \vdash S~\istype \\ 0\Gamma, x \stackrel0: S \vdash T~\istype}
  {0\Gamma \vdash (x \stackrel\rho: S) \to T~\istype}

  \inferrule*
  {\Gamma, x \stackrel{\sigma\rho}: S \vdash M \stackrel\sigma: T}
  {\Gamma \vdash \lambda x.M \stackrel\sigma: (x \stackrel\rho: S) \to T}

  \inferrule*
  {\Gamma_1 \vdash M \stackrel\sigma: (x \stackrel\rho: S) \to T \\
    \Gamma_2 \vdash N \stackrel{\sigma'}: S \\
    0\Gamma_1 = 0\Gamma_2 \\
    \sigma' = 0 \Leftrightarrow (\rho = 0 \lor \sigma = 0)}
  {\Gamma_1 + \rho\Gamma_2 \vdash M\,N \stackrel\sigma: T[N/x]}
\end{mathpar}
The side conditions on the elimination rule state that (i) both
$\Gamma_1$ and $\Gamma_2$ erase to the same context, so their sum is
defined; and (ii) the argument $N$ is erased (i.e., $\sigma' = 0$) iff
either the function does not use its argument, or we are in the
$\sigma = 0$ fragment and everything is being erased.

$\Sigma$-types are a little more involved, and demonstrate the
flexibility in QTT in allowing additional power in the $\sigma = 0$
fragment, where we do not need to care about
implementability. Formation and introduction are given by the rules:
\begin{mathpar}
  \inferrule*
  {0\Gamma \vdash S \\ 0\Gamma, x \stackrel0: S \vdash T}
  {0\Gamma \vdash (x \stackrel\pi: S) \otimes T}

  \inferrule*
  {\Gamma_1 \vdash M \stackrel{\sigma'}: S \\
    \Gamma_2 \vdash N \stackrel\sigma: T[M/x] \\\\
    0\Gamma_1 = 0\Gamma_2 \\
    \sigma' = 0 \Leftrightarrow (\pi = 0 \lor \sigma = 0)}
  {\pi\Gamma_1 + \Gamma_2 \vdash (M, N) \stackrel\sigma: (x \stackrel\pi: S) \otimes T}
\end{mathpar}
As with $\Pi$-typed, the first component of a $\Sigma$ type is
annotated with a usage for how many times it can be used, and this is
respected by the introduction rule. Elimination of $\Sigma$-types
depends on whether we are in the $\sigma = 0$ fragment or not. In the
$\sigma = 0$ fragment, we are free to disregard usage restrictions,
and use projections as normal:
\begin{mathpar}
  \inferrule*
  {\Gamma \vdash M \stackrel0: (x \stackrel\pi: S) \otimes T}
  {\Gamma \vdash \mathrm{fst}(M) \stackrel0: S}

  \inferrule*
  {\Gamma \vdash M \stackrel0: (x \stackrel\pi: S) \otimes T}
  {\Gamma \vdash \mathrm{snd}(M) \stackrel0: T[\mathrm{fst}(M)/x]}
\end{mathpar}
Hence, $\Sigma$-types act as normal in the $\sigma = 0$ fragment, and
we can use them as for normal type-theoretic constructions.  In the
$\sigma = 1$ fragment, we must take into account the resource content
of objects and use a pattern matching construct; the dependently typed
analogue of the $\otimes$-eliminator in \autoref{sec:lineartype}:
\begin{mathpar}
  \inferrule*
  {0\Gamma, z \stackrel0: (x \stackrel\pi: A) \otimes B \vdash C \\
    \Gamma_1 \vdash M \stackrel\sigma: (x \stackrel\pi: A) \otimes B \\
    \Gamma_2, x \stackrel{\sigma\pi}: A, y \stackrel{\sigma}: B \vdash N \stackrel\sigma: C[(x,y)/z]\\
    0\Gamma_1 = 0\Gamma_2}
  {\Gamma_1 + \Gamma_2 \vdash \mathrm{let}~(x,y) = M~\mathrm{in}~N \stackrel\sigma: C[M/z]}
\end{mathpar}

\subsubsection{The Identity Type}

QTT also supports an extensional equality type with equality reflection:
\begin{mathpar}
  \inferrule*
  {0\Gamma \vdash S~\istype \\
    0\Gamma \vdash M \stackrel0: S \\
    0\Gamma \vdash N \stackrel0: S}
  {0\Gamma \vdash M =_S N~\istype}

  \inferrule*
  {\Gamma \vdash M \stackrel\sigma: S}
  {\Gamma \vdash \conRefl(M) \stackrel\sigma: M =_S M}

  \inferrule*
  {\Gamma \vdash N \stackrel\sigma: M_1 =_S M_2}
  {\Gamma \vdash M_1 \equiv M_2 \stackrel\sigma: S}
\end{mathpar}
The equality type also has an $\eta$ rule demonstrating $\conRefl(M)$
as the only proof of equality \cite{hoffmann99dependent}.

\subsubsection{Universe} QTT can also be equipped in universe types,
just as in standard type theory. We refer to the rules presented in
\cite{atkey18qtt}.

\subsubsection{Datatypes} QTT, as we have presented it so far, has no
interesting base types to perform computation on. Following our
presentation of the simply typed linear systems in
\autoref{sec:linear-intro}, we add two kinds of datatype to QTT. First,
\autoref{sec:noniter-qtt}, we add non-iterable datatypes that allow
construction and case analysis, but no recursion. In
\autoref{sec:cons-free-qtt} we describe how to extend QTT to be a
dependently typed adaptation of the Cons-free system of
\autoref{sec:cons-free-intro}. In \autoref{sec:lfpl-qtt}, we apply the
same treatment to the LFPL-style system.

\subsection{Non-iterable Datatypes}
\label{sec:noniter-qtt}

\subsubsection{Booleans} The boolean type for QTT was described in
\cite{atkey18qtt}, but we repeat it here for completeness. Booleans
offer no possibility for iteration, but it is useful to see how the
QTT versions of the rules extend the simply typed rules from
\autoref{sec:linear-case-analysis} before moving to more complex
types.
\begin{mathpar}
  \inferrule*
  {\Gamma~\isctxt}
  {\Gamma \vdash \BoolTy~\istype}

  \inferrule*
  {\Gamma~\isctxt}
  {0\Gamma \vdash \cTrue, \cFalse \stackrel\sigma: \BoolTy}

  \inferrule*
  {0\Gamma_1, x \stackrel0: \BoolTy \vdash P~\istype \\
    \Gamma_1 \vdash M \stackrel\sigma: \BoolTy \\
    \Gamma_2 \vdash N_t \stackrel\sigma: P[\cTrue/x] \\
    \Gamma_2 \vdash N_f \stackrel\sigma: P[\cFalse/x] \\
    0\Gamma_1 = 0\Gamma_2}
  {\Gamma_1 + \Gamma_2 \vdash \If_{x.P}\: M \: \Then \: N_t \: \Else \: N_f \stackrel\sigma: P[M/x]}
\end{mathpar}
The introduction rules for booleans both use a $0$-d context,
indicating that construction of boolean values is free. Elimination of
booleans via a dependently typed if-then-else is more subtle with its
resource usage. The boolean to be eliminated must be constructed in a
context $\Gamma_1$, while the two branches are constructed in context
$\Gamma_2$. Since only one of the branches will be used, sharing
resources between the branches is expected.

One might wonder how, since constructing booleans is $0$-cost by their
introduction rules, the $\Gamma_1$ context will ever be non-$0$. This
is resolved by observing that booleans may be the output of processes
that consume time (e.g., the iteration constructs defined below), and
so $\Gamma_1$ will represent a requirement that the necessary resource
is provided.

\subsubsection{Lists} Lists are a little more complex than booleans,
because the $\cCons$ constructor takes two arguments, so their
resource usage must be combined. The type formation and introduction
rules are as follows:
\begin{mathpar}
  \inferrule*
  {0\Gamma \vdash T~\istype}
  {0\Gamma \vdash \ListTy(T)~\istype}

  \inferrule*
  {\Gamma \vdash T~\istype}
  {0\Gamma \vdash \cNil \stackrel\sigma: \ListTy(T)}

  \inferrule*
  {\Gamma_1 \vdash M \stackrel\sigma: T \\
    \Gamma_2 \vdash N \stackrel\sigma: \ListTy(T) \\
    0\Gamma_1 = 0\Gamma_2}
  {\Gamma_1 + \Gamma_2 \vdash \cCons(M,N) \stackrel\sigma: \ListTy(T)}
\end{mathpar}
Lists do have the potential for iteration by their recursive nature,
but in order to ensure the complexity guarantees we only permit
matching without recursion in the full calculus. Here is the rule for
dependently typed case analysis on lists:
\begin{mathpar}
  \inferrule*
  {0\Gamma_1, x \stackrel0: \ListTy(T) \vdash P~\istype \\
    \Gamma_1 \vdash M \stackrel\sigma: \ListTy(T) \\
    \Gamma_2 \vdash N_1 \stackrel\sigma: P[\cNil/x] \\
    \Gamma_2, h \stackrel\sigma: T, t \stackrel\sigma: \ListTy(T) \vdash N_2 \stackrel\sigma: P[\cCons(h,t)/x] \\
    0\Gamma_1 = 0\Gamma_2}
  {\Gamma_1 + \Gamma_2 \vdash \Match_{x.P}\,M\,\{\,\cNil \mapsto N_1; \cCons(h,t) \mapsto N_2\,\} : P[M/x]}
\end{mathpar}
In the $\sigma = 0$ fragment, however, we are free to iterate on lists
because computations in this fragment are only meant for type-level
computation, not for the program itself. Put another way, the type
checker may perform arbitary recursion on lists to type check the
program, but the program itself may not do so without correctly
accounting its costs as described in the following sections. The
$\sigma = 0$ recursor for lists has the following typing rule, which
is the standard dependent eliminator for lists, with everything
annotated as $0$ usage.
\begin{mathpar}
  \inferrule*
  {0\Gamma, x \stackrel0: \ListTy(T) \vdash P~\istype \\
    0\Gamma \vdash M \stackrel0: \ListTy(T) \\
    0\Gamma \vdash N_1 \stackrel0: P[\cNil/x] \\
    0\Gamma, h \stackrel0: T, t \stackrel0: \ListTy(T), p \stackrel0: P[t/x] \vdash N_2 \stackrel0: P[\cCons(h,t)/x]}
  {0\Gamma \vdash \mathrm{recList}_{x.P}\,M\,\{\,\cNil \mapsto N_1; \cCons(h,t;p) \mapsto N_2\,\} \stackrel0: P[M/x]}
\end{mathpar}

\subsection{Cons-free Natural Numbers and their Recursor}
\label{sec:cons-free-qtt}

The datatypes of the previous section still only allow us to write
programs in the $\sigma = 1$ fragment that are constant time in the
size of their input. As with the simply typed linear system, if we are
handed a list of an unknown length, we can only explore it to a fixed
depth determined statically. To write programs that do work
proportional to the size of their input, we need some form of iterable
datatype. In both our Cons-free and LFPL-style QTT systems, we use a
natural number datatype.

The cons-free system cannot allow the programmer to construct natural
numbers in the $\sigma = 1$ fragment, as this would violate the
complexity guarantees. However, we can use the flexibility of QTT to
allow free construction of naturals in the $\sigma = 0$ fragment,
which allows us to use natural numbers freely in types. Therefore, we
have the following introduction rules, only usable in the $\sigma = 0$
fragment:
\begin{mathpar}
  \inferrule*
  {\Gamma \vdash}
  {\Gamma \vdash \conZero \stackrel0: \tyNat}

  \inferrule*
  {\Gamma \vdash M \stackrel0: \tyNat}
  {\Gamma \vdash \conSucc(M) \stackrel0: \tyNat}
\end{mathpar}
The cons-free system allows free duplication of complete natural
numbers. This is accomplished by a special construct copying the
simply linear typed rule we gave above:
\begin{mathpar}
  \inferrule*
  {\Gamma \vdash M \stackrel\sigma: \tyNat}
  {\Gamma \vdash \mathrm{dupNat}(M) \stackrel\sigma: \tyNat \otimes \tyNat}
\end{mathpar}
Anyone who has reasoned about the metatheory of or implemented a type
checker for dependent types will view this rule with unease as it
appears to grant the ability to construct non-canonical values of pair
type, and consequently generate non-canonical naturals. We fix this by
adding an equational rule to the $\sigma = 0$ fragment, ensuring
\emph{definitionally} that $\mathrm{dupNat}$ acts as its name implies:
\begin{mathpar}
  \inferrule*
  {\Gamma \vdash M \stackrel0: \tyNat}
  {\Gamma \vdash \mathrm{dupNat}(M) \equiv (M,M) \stackrel0: \tyNat \otimes \tyNat}
\end{mathpar}
Note that this rule is well-typed by the $0$-needs-$0$ property of
QTT, and the fact that $0+0=0$.

The eliminator for these natural numbers takes the following
form. Disregarding the usage annotations, this has the same type
structure as the normal dependently typed recursor for natural
numbers:
\begin{displaymath}
  \mprset{flushleft}
  \inferrule*
  {0\Gamma, x \stackrel0: \tyNat \vdash P~\istype \\\\
    \Gamma \vdash M \stackrel\sigma: \tyNat \\\\
    0\Gamma \vdash N_z \stackrel\sigma: P[\conZero/x] \\\\
    0\Gamma, n \stackrel0: \tyNat, p \stackrel\sigma: P[n/x] \vdash N_s \stackrel\sigma: P[\conSucc(n)/x]}
  {\Gamma \vdash \tmRec_{x.P}\,M\,\{\conZero \mapsto N_z; \conSucc(n;p) \mapsto N_s\} \stackrel\sigma: P[M/x]}
\end{displaymath}
In the successor case, $N_s$, there are two bound variables: $n$ for
the natural number and $p$ for its induction hypothesis. Note that $n$
is required to be usage $0$ no matter what $\sigma$ is. We need the
variable $n$ to be present in order to correctly type the induction
hypothesis and the conclusion, but it must be marked as usage $0$ to
ensure that the resources captured by the number are not duplicated.

The reader is invited to compare this dependently typed rule with the
simply typed linear version in \autoref{sec:cons-free-intro}. Removing
the $0$-annotated parts of the rule, and the type dependency, yield
the exact same rule. Conversely, when $\sigma = 0$, this rule is
identical (up to $0$-annotations) to the usual dependently typed
recursor for natural numbers, and so we can use it in the types to
prove properties of programs just as we do in normal type theory. We
will see in \autoref{sec:cons-free-sound} that this rule is realisable
by polynomial-time amortised computation, and so is sound for
polynomial time.

\subsection{LFPL-style Diamonds and Natural Numbers}
\label{sec:lfpl-qtt}

As explained in \autoref{sec:lfpl-intro}, the LFPL system differs from
the Cons-free system in that it is possible to construct natural
numbers (and other iterable datatypes), provided one has the necessary
diamonds to pay for the construction. As with the natural number type
in the Cons-free system, it ought not be possible to construct
diamonds in the $\sigma = 1$ fragment, as this would amount to the
free distribution of diamonds to all which would lead to a collapse in
the complexity guarantees of the system. It is possible construct
diamonds in the $\sigma = 0$, though:
\begin{mathpar}
  \inferrule*
  {\Gamma \vdash}
  {0\Gamma \vdash \Diamond~\istype}

  \inferrule*
  {\Gamma \vdash}
  {0\Gamma \vdash * \stackrel0: \Diamond}

  \inferrule*
  {\Gamma \vdash M \stackrel0: \Diamond}
  {\Gamma \vdash M \equiv * \stackrel0: \Diamond}
\end{mathpar}
The $\Diamond$ type also supports an $\eta$-rule in the $\sigma = 0$
fragment, indicating that, in this fragment, it acts the same as a
unit type. This allows us to use freely use diamonds in types, and to
not have to care about the identity of particular diamonds, since by
this rule all diamonds are definitionally
equal\footnote{\emph{Fungible}, if one wishes to use a monetary
  metaphor.}.
% Link to Selinger et al: the $\eta$ rule for $\Diamond$s means that are
% essentially depending on the ``shape'' of anything that depends on
% diamonds.

Construction of natural numbers now requires a $\Diamond$ for
$\conZero$ and a $\Diamond$ and a predecessor for $\conSucc$:
\begin{mathpar}
  \inferrule*
  {\Gamma \vdash M \stackrel\sigma: \Diamond}
  {\Gamma \vdash \conZero(M) \stackrel\sigma: \tyNat}

  \inferrule*
  {\Gamma_1 \vdash M \stackrel\sigma: \Diamond \\
    \Gamma_2 \vdash N \stackrel\sigma: \tyNat \\
    0\Gamma_1 = 0\Gamma_2}
  {\Gamma_1 + \Gamma_2 \vdash \conSucc(M,N) \stackrel\sigma: \tyNat}
\end{mathpar}
In the $\sigma = 0$ fragment, we can construct $\Diamond$s for free,
and so construct natural numbers freely as well just as we did for the
Cons-free system above.

The dependently typed recursor for LFPL-style natural numbers again
augments the simply typed linear recursor from
\autoref{sec:lfpl-intro} with dependency information:
\begin{mathpar}
  \mprset{flushleft}
  \inferrule*
  {0\Gamma, x \stackrel0: \tyNat \vdash P~\istype \\\\
    \Gamma \vdash M \stackrel\sigma: \tyNat
    0\Gamma, d \stackrel1: \Diamond \vdash N_z \stackrel\sigma: P[\conZero(*)/x] \\\\
    0\Gamma, d \stackrel1: \Diamond, n \stackrel0: \tyNat, p \stackrel1: P[n/x] \vdash N_s \stackrel\sigma : P[\conSucc(*,n)/x]}
  {\Gamma \vdash \tmRec\,M\,\{\conZero(d) \mapsto N_z; \conSucc(d,n;p) \mapsto N_s\} \stackrel\sigma: P[M/x]}
\end{mathpar}
We have used $* : \Diamond$ as the value in the types for the zero and
successor cases. By the $\eta$-rule for diamonds, we could have
equally well used the $d$ variable that is in scope in each case.

Just as for the Cons-free system iterator above, in the $\sigma = 0$
fragment this rule is identical to the usual dependently typed
recursor for the natural numbers, so it can be used in the types to
reason about programs. Moreover, we will see in
\autoref{sec:lfpl-sound} that this rule is also sound for polynomial
time in a system with $\Diamond$s.

\subsection{Reflection of Realisability}
\label{sec:qtt-reflection}

Our final addition to QTT is \emph{reflection of realisability}. In
QTT thus far, it has been possible to reason about the non-resourced
behaviour of programs. This is because the $0$-ing process moving from
the $\sigma = 1$ fragment to the $\sigma = 0$ fragment erases all
resource information. This is sufficient for reasoning about the
extensional behaviour of programs, but it is useful to be able to make
statements like ``this function is realisable in polynomial time'' in
the types of QTT, something that is not current possible.

We remedy this by adding a \emph{realisable} type to QTT with the
following type formation and introduction and elimination rules:
\begin{mathpar}
  \inferrule*
  {0\Gamma \vdash A~\istype}
  {0\Gamma \vdash \Rtype(A)~\istype}

  \inferrule*
  {0\Gamma \vdash M \stackrel1: A}
  {0\Gamma \vdash \rIntro(M) \stackrel\sigma: \Rtype(A)}

  \inferrule*
  {\Gamma \vdash M \stackrel\sigma: \Rtype(A)}
  {\Gamma \vdash \rElim(M) \stackrel{\sigma'}: A}
\end{mathpar}
Intuitively, the type $\Rtype(A)$ is inhabited whenever the type $A$
is realisable in the $\sigma = 1$ fragment of the system. In
particular, the type $\Rtype(\tyNat \to \tyNat)$ is the type of all
realisable functions from natural numbers to natural numbers. In the
polynomial time systems we are concerned with here, this is exactly
the type of polynomial functions. Note that in the introduction rule,
the premise is required to be in the $\sigma = 1$ fragment, to ensure
that the type is realisable, while in the elimination rule, the
conclusion is in an arbitrary fragment $\sigma'$. This flexibility is
require to maintain the admissibility of the $0$-ing rule.

With just the rules given here, the type $\Rtype(A)$ is nothing more
than a statement that a given type is realisable with a polytime
implementation. This is enough to do some proofs that we present in
the next section, e.g., that polytime functions are closed under
composition, but one could imagine stronger reflection principles that
allow deeper logical consequences of polytime realisability to be
proved internally. We discuss this further in
\autoref{sec:towards-synthetic-complexity-theory}.

Readers familiar with Benton's Linear/Non-linear system
\cite{benton94} will note that the $\Rtype(A)$ constructor is the QTT
analogue of the right adjont $G$ type constructor in that system. The
$\Sigma$-types describe above play the role of the left adjoint $F$
types, in a similar way to the dependent linear type system of Pradic,
Krishnaswami and Benton \cite{pradic}.

%%%%%%%%%%%%%%%%%%%%%%%%%%%%%%%%%%%%%%%%%%%%%%%%%%%%%%%%%%%%%%%%%%%%%%
%%%%%%%%%%%%%%%%%%%%%%%%%%%%%%%%%%%%%%%%%%%%%%%%%%%%%%%%%%%%%%%%%%%%%%
%%%%%%%%%%%%%%%%%%%%%%%%%%%%%%%%%%%%%%%%%%%%%%%%%%%%%%%%%%%%%%%%%%%%%%
%%%%%%%%%%%%%%%%%%%%%%%%%%%%%%%%%%%%%%%%%%%%%%%%%%%%%%%%%%%%%%%%%%%%%%
\section{Programming and Proving with Polytime}
\label{sec:programming-polytime}

\begin{enumerate}
\item Non-deterministic polytime via a bit oracle -- needs a statement
  of success
\item Previous could also be interpreted as probabilistic polytime
  (access to coin flips); can we formalise some kind of cryptographic
  property this way? Have a look at hash functions in the
  Authenticated Data Structures paper.
\item Insertion sort, quicksort as polytime functions.
\item A type of polytime functions; via a reflection operator
\item Programming with explicit resource annotations, including
  reasoning about the resource requirements directly.
\end{enumerate}

%%%%%%%%%%%%%%%%%%%%%%%%%%%%%%%%%%%%%%%%%%%%%%%%%%%%%%%%%%%%%%%%%%%%%%
%%%%%%%%%%%%%%%%%%%%%%%%%%%%%%%%%%%%%%%%%%%%%%%%%%%%%%%%%%%%%%%%%%%%%%
%%%%%%%%%%%%%%%%%%%%%%%%%%%%%%%%%%%%%%%%%%%%%%%%%%%%%%%%%%%%%%%%%%%%%%
%%%%%%%%%%%%%%%%%%%%%%%%%%%%%%%%%%%%%%%%%%%%%%%%%%%%%%%%%%%%%%%%%%%%%%
\section{Polytime Soundness via Realisability}
\label{sec:soundness}

\newcommand{\cstaccess}{c_{\mathit{access}}}
\newcommand{\cstmkclo}{c_{\mathit{mkclo}}}
\newcommand{\cstapp}{c_{\mathit{app}}}
\newcommand{\cstmkpair}{c_{\mathit{mkpair}}}
\newcommand{\cstmkunit}{c_{\mathit{mkunit}}}
\newcommand{\cstTrue}{c_{\mathit{mktrue}}}
\newcommand{\cstFalse}{c_{\mathit{mkfalse}}}
\newcommand{\cstLetpair}{c_{\mathit{letpair}}}
\newcommand{\cstSeq}{c_{\mathit{seq}}}
\newcommand{\cstIf}{c_{\mathit{if}}}

\newcommand{\clo}[2]{\mathsf{clo}\langle #1 , #2 \rangle}
\newcommand{\synTrue}{\mathsf{true}}
\newcommand{\synFalse}{\mathsf{false}}

\newcommand{\ExpSet}{\mathcal{E}}
\newcommand{\ValSet}{\mathcal{V}}

\newcommand{\rplus}{\oplus}
\newcommand{\rzero}{\emptyset}

\bob{introductory paragraph.}

The Hofmann-Dal Lago approach to proving the soundness of implicit
computational complexity systems is based on a three way coupling
between abstract mathematical elements (the \emph{what}), values from
a machine model (the \emph{how}), and resource potentials (the
\emph{fuel}). Each type in the system is defined as a three way
relation between these elements. The set of abstract elements depends
on the type being interpreted (e.g., types of natural numbers will be
defined in terms of the set $\mathbb{N}$). The machine model is fixed
across all types. We describe the particular machine model we use for
this paper in \autoref{sec:machine-model}. Potentials are arranged
into \emph{resource monoids} that we define in
\autoref{sec:resource-monoids}. Following Hofmann and Dal Lago, we
will select resource monoids appropriate to the kind of system that we
want to prove complexity soundness for.

\subsection{Machine Model and Operational Semantics}
\label{sec:machine-model}

\begin{figure}
  \centering
  {\bf Syntax}
  \begin{displaymath}
    \begin{array}{lcl}
      i,j &\in&\mathbb{N} \\
      E \in \ExpSet &::=& \lambda E \mid * \mid (i, j) \mid \synTrue \mid \synFalse \mid i \mid \Let\,E_1\,\In\,E_2 \mid i \cdot j \mid \LetPair\,i\,\In\,E \mid \If\,i\,E_1\,E_2 \\
      V \in \ValSet &::=& \clo{E}{\eta} \mid * \mid (V_1, V_2) \mid \synTrue \mid \synFalse \\
      \eta &::= & [] \mid \eta :: V
    \end{array}
  \end{displaymath}

  \vspace{1em}

  {\bf Evaluation: Construction}
  \begin{mathpar}
    \inferrule* [right=MkClo]
    { }
    {\lambda E , \eta \Downarrow_{\cstmkclo} \clo{E}{\eta}}

    \inferrule* [right=MkUnit]
    { }
    {*, \eta \Downarrow_{\cstmkunit} *}

    \inferrule* [right=MkPair]
    {\eta[i] = V_1 \\ \eta[j] = V_2}
    {(i, j), \eta \Downarrow_{\cstmkpair} (V_1, V_2)}

    \inferrule* [right=MkTrue]
    { }
    {\synTrue, \eta \Downarrow_{\cstTrue} \synTrue}

    \inferrule* [right=MkFalse]
    { }
    {\synFalse, \eta \Downarrow_{\cstFalse} \synFalse}
  \end{mathpar}

  \vspace{1em}

  {\bf Evaluation: Variable access and Sequencing}
  \begin{mathpar}
    \inferrule* [right=Access]
    {\eta[i] = v}
    {i, \eta \Downarrow_{\cstaccess} v}

    \inferrule* [right=Seq]
    {E_1, \eta \Downarrow_{k_1} V \\
      E_2, (\eta :: V) \Downarrow_{k_2} V'}
    {\Let\,E_1\,\In\,E_2, \eta \Downarrow_{k_1 + \cstSeq + k_2} V'}
  \end{mathpar}

  \vspace{1em}

  {\bf Evaluation: Elimination}
  \begin{mathpar}
    \inferrule* [right=App]
    {\eta[i] = \clo{E}{\eta'} \\
      \eta[j] = V \\
      E , (\eta' :: \clo{E}{\eta'} :: V) \Downarrow_k V'}
    {(i \cdot j), \eta \Downarrow_{\cstapp + k} V'}

    \inferrule* [right=LetPair]
    {\eta[i] = (V_1, V_2) \\
      E, (\eta :: V_1 :: V_2) \Downarrow_k V}
    {\LetPair\,i\,\In\,E, \eta \Downarrow_{\cstLetpair + k} V}

    \inferrule* [right=IfTrue]
    {\eta[i] = \synTrue \\ E_1, \eta \Downarrow_k V}
    {\If\,i\,E_1\,E_2, \eta \Downarrow_{\cstIf + k} V}

    \inferrule* [right=IfFalse]
    {\eta[i] = \synFalse \\ E_2, \eta \Downarrow_k V}
    {\If\,i\,E_1\,E_2, \eta \Downarrow_{\cstIf + k} V}
\end{mathpar}
  \caption{Language with CBV Big-step Costed Evaluation Semantics}
  \label{fig:opsem}
\end{figure}

Note: the Agda formalisation is intrinsically well-scoped, which
makes expressing things a bit easier.

The syntax and rules of our target language are given in
\autoref{fig:opsem}.

\begin{enumerate}
\item Justification that looking up variables is constant: Due to
  lexical scoping, the size of the maximum variable lookup is bounded
  for a fixed program.
\item An advantage of our more ``elementary'' operational semantics
  (rather than encoding everything in the $\lambda$-calculus)
\end{enumerate}

\subsection{Resource Monoids}
\label{sec:resource-monoids}

Resource potentials are attached to values to represent the amount of
intrinsic potential they have to fuel computation. Resource potentials
are organised into resource monoids. To motivate the definition, we
first enumerate the structure we will. To be able to account for the
combined potential attached to composite data and programs (e.g.,
pairs, or functions applied to arguments) we will require monoid
structure on potentials. The action of turning potential difference
into fuel for computation will be modelled by a difference
function. Finally, we require that our resource monoid contains
sufficient elements to fuel constant time operations. We gather these
requirements into a formal definition as follows:

\begin{definition}
  A \emph{resource monoid} $M$ consists of:
  \begin{enumerate}
  \item A carrier set $|M|$, whose elements represent amounts of
    potential. We use Greek letters $\alpha$, $\beta$, $\gamma$ to
    denote elements of a resource monoid.
  \item Commutative monoid structure $(\rplus,\rzero)$ on $|M|$, so
    we can add potentials.
  \item A \emph{difference function} $M : |M| \times |M| \to \natinf$,
    where $\natinf$ is the natural numbers extended with a negative
    infinity $- \infty$ and $- \infty + k = -\infty$. A difference
    $M(\alpha, \beta) = k \in \mathbb{N}$ means that starting with
    potential $\alpha$ and ending with potential $\beta$ yields $k$
    units of fuel. A difference of $- \infty$ means that $\alpha$
    contains insufficient potential to reach $\beta$. Differencing
    must satisfy:
    \begin{enumerate}
    \item for all $\alpha$, $M(\alpha,\alpha) = 0$; and
    \item for all $\alpha, \beta, \gamma$,
      $M(\alpha, \beta) \rplus M(\beta, \gamma) \leq M(\alpha, \gamma)$.
    \end{enumerate}
    The latter is a ``reverse triangle inequality'': the fuel
    recoverable by moving between potential levels $\alpha$ and
    $\gamma$ via $\beta$ may be less than the fuel recoverable
    moving from $\alpha$ to $\gamma$ directly.
  \item \bob{Interaction between $(\rplus,\rzero)$ and $M(\alpha,\beta)$.}
  \item An \emph{accounting function}
    $\mathit{acct} : \mathbb{N} \to |M|$ such that for all $k$,
    $k \leq M(\mathit{acct}(k),\rzero)$.
  \end{enumerate}
\end{definition}

\subsubsection{Specific Resource Monoids}

The simplest example of a resource monoid is given by the natural
numbers $\mathbb{N}$, where each number stands directly an amount of
stored fuel.

\begin{definition}[Natural Number Resource Monoid]
  Monoid structure is given by normal addition. Differencing is
  defined as
  \begin{displaymath}
    \mathbb{N}(m,n) = \left\{
      \begin{array}{ll}
        m - n & m \geq n \\
        - \infty & \textrm{otherwise}
      \end{array}\right.
  \end{displaymath}
  and $\mathit{acct}(k) = k$. Note that this is the simplest possible
  resource monoid due to the requirement that the $\mathit{acct}$
  function exists. \bob{Forward ref to where this is used.}
\end{definition}

Every resource monoid induces a partial ordering on its carrier set by
$\alpha \leq \beta$ iff $0 \leq M(\alpha, \beta)$. In the case of the
natural number resource monoid, this yields the usual ordering on the
naturals.

\newcommand{\MaxPoly}{\mathrm{MaxPoly}}
\newcommand{\PlusPoly}{\mathrm{PlusPoly}}

The differencing operator of the natural number resource monoid can
only supply as much fuel as is contained in the potential. This makes
it suitable for amortised analysis, where we directly store potential
in data structures. For the two polynomial time systems, we need more
sophisticated structures originally presented by Dal Lago and
Hofmann. The fundamental idea with both is to represent potentials as
pairs $(m,p)$, where $m$ is a natural number and $p$ is a
polynomial. The $m$ tracks the ``size'' of data as it pertains to the
number of times an operation will be repeated by iterating over it ---
for example, an iterable natural number will have size equal to
itself, but a non-iterable natural number may be assigned zero
size. The polynomial $p$ tracks the complexity of a program as a
function of the size of the input. This leads to a differencing
operator that evaluates the polynomial with the size of the data:

\begin{definition}[Polynomial Resource Monoids]
  The \emph{Max-Polynomial} resource monoid $\MaxPoly$ has carrier
  set consisting of pairs $(m,p)$ where $m$ is natural number and $p$
  is a polynomial with natural number coefficients. Addition of
  elements is defined as $(m,p) \oplus (n,q) = (m \sqcup n, p+q)$ with
  $\emptyset = (0,0)$. Difference is defined as:
  \begin{displaymath}
    \MaxPoly((m,p),(n,q)) = \left\{
      \begin{array}{ll}
        p(m) - q(m)&m \geq n \textrm{ and } \forall k \geq m. p(k) \geq q(k) \\
        - \infty & \textrm{otherwise}
      \end{array}
    \right.
  \end{displaymath}
  $\MaxPoly$ accounts for constant time with constant polynomials:
  $\mathit{acct}(k) = (0,\lambda x.k)$.

  The \emph{Plus-Polynomial} resource monoid $\PlusPoly$ is defined
  the same way as $\MaxPoly$ except that the monoid addition adds the
  natural number components instead of taking their maximum:
  $(m,p) \oplus (n,q) = (m + n, p+q)$.
\end{definition}

It is perhaps easier to see how the differencing operator works in the
special case of the difference $\MaxPoly((m,p),(0,0)) = p(m)$. I.e.,
if we have code that contains data of size $m$ and a program with
complexity $p$, then running the combination with no expectation of
remaining potential yields $p(m)$ available steps.

The $\MaxPoly$ and $\PlusPoly$ resource monoids will be used for the
cons-free and LFPL-style (i.e., with diamonds) systems
respectively. We will elaborate on how in
\autoref{sec:realisability-model} and how they support the two kinds
of natural number iteration in \autoref{sec:realising-iteration}.

FIXME: difference between this definition and Hofmann-Dal Lago's: we
use polynomials with natural number coefficients instead of a function
that is bounded by a polynomial.

\subsubsection{Resource sub-monoids}

The separation between sizes of data and complexity of code in the
polynomial resource monoids motivates the use of resource sub-monoids
to ensure that programs themselves (as opposed to higher order code
which may contain closed over data) do not contain data that can be
iterated. We do this by requiring that programs' potential must come
from a specified resource sub-monoid:

\begin{definition}[Resource Sub-Monoids]
  A \emph{resource sub-monoid} $M_0 \subseteq M$ of a resource monoid
  $M$ consists of a subset $|M_0| \subseteq |M|$ that is closed under
  the monoid operations and $\mathit{acct}$.
\end{definition}

For both $\MaxPoly$ and $\PlusPoly$, the elements with zero size
component, i.e., of the form $(0,p)$, form a resource sub-monoid that
we will use for interpreting programs. We will call these sub-monoids
$\MaxPoly_0$ and $\PlusPoly_0$. We make use of resource sub-monoids
in our definitions of the interpretations of simultaneous
substitutions (\autoref{defn:rl-morphism}) and terms
(\autoref{defn:rl-terms}).

\subsubsection{Resource Monoids as Enriched Categories}

\bob{move this to below the definition of resource monoid}

An alternative perspective on resource monoids is to see them as a
kind of entriched category \cite{kelly}. Enriched categories replace
the sets of morphisms between objects with objects from some category
other than $\Set$. Define the posetal category $\natinf$ with objects $\mathbb{N} \cup \{\infty\}$ and

$\natinf$ as a category.

\bob{remember to look at the Marsden and thingy paper about
  quantitative resources and enriched categories.}  A concise
definition of Hofmann and Dal Lago's resource monoids can be given as:
\begin{definition}
  A \emph{resource monoid} is a strict symmetric monoidal
  $\natinf$-enriched category.
\end{definition}

\subsection{Models of Quantitative Type Theory}
\label{sec:qtt-models}

\bob{PLAN: RCwFs, and state that they can be constructed from Indexed Linear Posets}

\cite{atkey18} describes a general class of QTT models termed
\emph{Quantitative Categories with Families} (QCwFs). This class of
models fits will with the syntax of QTT, but to actually build models
it is revealing to construct them from certain indexed partial orders,
as we explain now. In the next section, we will construct a class of
specific models that prove the soundness of our complexity constrained
systems.

\begin{definition}[Category with Families]
  BLAH
\end{definition}

\begin{definition}
  A \emph{linear preorder} is a preordered set $(A, \leq)$ with
  \begin{enumerate}
  \item A commutative monoid $(I, -\otimes-)$ that is monotone
    w.r.t. the order; and
  \item Is closed: there is an operation
    $\multimap : A \times A \to A$ such that $x \otimes y \leq z$ iff
    $x \leq y \multimap z$.
  \item Right adjoint
  \end{enumerate}
  The collection of all linear preorders and functions that preserve the
  order and the operations forms a category $\LinPreorder$.
\end{definition}

\begin{definition}
  \begin{enumerate}
  \item A CwF $(\cat{C}, \top, \langle-,-\rangle, ...)$ that supports
    $\Pi$ and $\Sigma$ types.
  \item A functor $L : \cat{C}^\op \to \LinPreorder$, such that
    reindexing along projections has a right adjoint:
    \begin{displaymath}
      L_{\Gamma.A}(\pi^*X, Y) \cong L_{\Gamma}(X, \forall_A Y)
    \end{displaymath}
  \end{enumerate}
\end{definition}

\bob{define the construction}
\begin{enumerate}
\item Start with a CwF $\cat{C}$ that supports $\Pi$, $\Sigma$ and
  natural numbers.
\item Assume a functor $L : \cat{C}^\top \to \LinPreorder$ with
  reindexing along projections having a right adjoint.
\item Construct a new CwF with the same base category as $\cat{C}$,
  but with $\Ty'(\Delta) = \Sigma A : \Ty(\Delta).~L(\Delta.A)$ and
  $\Tm'(\Delta, (T, X)) = \Tm(\Delta, T)$.
\item
\end{enumerate}

This construction is implicit in \cite{atkey2018qtt}, where the
indexed linear poset is defined from a $\mathcal{R}$-LCA.


\subsection{Amortised Complexity Realisability Model}
\label{sec:realisability-model}

Equipped with our underlying costed model of computation
(\autoref{sec:machine-model}) and a compositional notion of resource
potential (\autoref{sec:resource-monoids}), we construct models of QTT
that witness the resource and type soundness of our complexity
constrained systems.

\bob{Tie back to previous section}

We fix a resource monoid $M$ with distinguished sub-monoid $M_0$.

\subsubsection{Base CwF}
\label{sec:realisability-base-cwf}

As our base CwF, we use the category $\Set$ of sets and functions with
the standard interpretation of $\Pi$, $\Sigma$ types and the
$\mathbb{N}$ type.

\subsubsection{Indexed Linear Preorder}
\label{sec:realisability-indexed-linear-preorder}

We now define an indexed linear poset $L$ of realisers over $\Set$
that ties together our ``mathematical'' model of types in $\Set$ with
our machine model and resource monoid. For a set $A$, the carrier of
$L(A)$ is the set of ternary relations
$X \subseteq A \times M \times \ValSet$ and we define the ordering
$X \leq Y$ to hold iff there exists realising expression
$E \in \ExpSet$ and potential $\gamma \in M_0$ such that for all
$a \in A$, $\alpha \in M$ and $v \in \ValSet$ with
$(a,\alpha,v) \in X$, we have that there exists a result
$v' \in \ValSet$, step count $k \in \mathbb{N}$ and result potential
$\beta \in M$ with:
\begin{enumerate}
\item $E, v \Downarrow_k v'$ (evaluation successfully completes in $k$ steps);
\item $(a, \beta, v') \in Y$ (the result is well-resourced and
  satisfies $Y$); and
\item $k \leq M(\alpha \rplus \gamma, \beta)$ (the step count is within the
  difference between the initial potential and the result potential).
\end{enumerate}
Note that the definition of realisablity is uniform in the element $a$
-- the realising expression $e$ and the potential $\gamma$ must work
for all $a$ -- thus the implementation and complexity measure of the
transition being modelled cannot depend on what the input is. Put in
implementation terms, the input $a$ is not present at
runtime. Moreover note that the potential $\gamma$ attached to the
expression $e$ must come from the sub-monoid $M_0$, indicating that is
intended to be data-free, while the potential $\alpha$ for the input
is from the full monoid $M$, so it can contain data and functions.

For $X, Y \in L(A)$, the required elements for symmetric monoidal
closed structure are defined as follows. For the tensor product
$X \otimes Y \in L(A)$, the realising value must be a pair $(v_1,v_2)$
and the potential of the pair must split into suitable potentials
$\alpha_1$, $\alpha_2$ for the components. For the residual
$X \multimap Y$, the realising value must be a closure with potential
to, when added to the potential of an input, compute the output with
enough remaining. Note that the potential attached to a closure
($\alpha$, here) need not be from the sub-monoid $M_0$. Unlike
top-level term interpretations, closures may contain data.
\begin{displaymath}
  \begin{array}{lcl}
    X \otimes Y &=& \{ (a, \alpha, (v_1, v_2)) \mid \exists \alpha_1, \alpha_2.~0 \leq M(\alpha, \alpha_1 \rplus \alpha_2) \land X(a,\alpha_1,v_1) \land Y(a,\alpha_2,v_2) \}\\
    X \multimap Y &=& \{ (a, \alpha, \clo{E}{\eta}) \mid
                      \begin{array}[t]{@{}l}
                        \forall \alpha' \in M, v,w \in \ValSet.\,X(a,\alpha',v) \Rightarrow\\
                        \quad \exists v', k, \beta.\,
                        E, (\eta :: w :: v) \Downarrow_k v' \land Y(a,\beta,v') \land k \leq M(\alpha \rplus \alpha', \beta) \} \end{array}
  \end{array}
\end{displaymath}
The seemingly useless $w \in \ValSet$ in the formula for
$X \multimap Y$ is a dummy argument standing for the self-referential
reference to the closure used for defining recursive programs.

Each $L(A)$ has a terminal (i.e. top) element, which is also the unit
for $\otimes$, defined as
$I_A = \{(a, \alpha, *) \mid a \in A, \alpha \in M\}$. The potential
$\alpha$ here is unrestricted, so $I_A$ can consume an arbitrary
resource.

$\mathbb{N}$-Graded exponentials in each $L(A)$ are defined using the
action $(\cdot)$ of $(\mathbb{N}, \leq)$ on $M$ defined above \bob{do
  this}. At the level of realising values, the modality $\oc_n$ has no
effect; it only serves to alter the required potentials.
\begin{displaymath}
  \oc_n\,X = \{(a,\alpha,v) \mid \exists \alpha'.\,M(n \cdot \alpha', \alpha) = 0 \land (a, \alpha', v) \in X \}
\end{displaymath}

$L$ also has arbitrary $\Set$-indexed products, realised ``lazily'' as
functions that take dummy arguments. For $A \in \Set$ and
$B \in A \to \Set$ and $X \in L(\Sigma A.\,B)$, we define
$\forall_B X \in L(A)$ similarly to $\multimap$ above, but with
different resource and indexing requirements:
\begin{displaymath}
  \forall_B X = \{ (a,\alpha,\clo{E}{\eta}) \mid \forall b, v.~\exists v', \beta, k. E, (\eta :: v :: *) \Downarrow_k v' \land X((a,b),\beta,v') \land k \leq M(\alpha,\beta) \}
\end{displaymath}
Note, as with the definition of $X \leq Y$ above, the realiser closure
$\clo{E}{\eta}$ must be chosen uniformly for all $b$. This definition
also appears to allow arbitrary computation (paid for by $\alpha$) to
happen when the realising closure is applied, but the potential
$\alpha$ will only ever be greater than $\beta$ by enough to handle
the administrative costs of applying the function.

To complete the construction of $L$ as an indexed linear preorder, we
need to give realisers for each of the required inequalities in
\autoref{def:indexed-linear-preorder}. In each case, this is a matter
of programming in the language of \autoref{sec:machine-model}. The
corresponding potentials are calculated by counting the steps in the
ensuing programs. Note that, so far, all realisers are constant time
(relative to their input), so the constructions so far work for any
resource monoid. For systems that require iteration, we will need more
structure on our resource monoids \bob{fwd ref}.

\begin{proposition}
  $L$, with $I$, $\otimes$, $\multimap$, $\oc_n$, and $\forall_{B}$ defined above, is
  an indexed linear preorder.
\end{proposition}

\begin{proof}
  See the Agda development.
\end{proof}

\subsubsection{Non-iterable Data Types}

The model of QTT constructed in
\autoref{prop:basic-qtt-realisability-model} does not yet include any
useful base types. Handling iterable types, which are the ones that
induce non-constant-time complexities, requires specific properties of
resource monoids that we introduce in
\autoref{sec:realising-iteration}.

Rather than perform a complex generic construction of datatypes to
show how non-iterable versions may be realised, we focus on two
representative examples. Booleans are the simplest case, with only two
cases and no chance of iteration. Lists are more complex; we can have
non-iterable lists containing iterable data. Combining lists and
booleans allows us to represent other datatypes, such as the
$\TapeType$ type used to represent Turing machine tapes in
\autoref{sec:completeness}.

\paragraph{Booleans} Fix $\mathbb{B} = \{ \mathit{tt}, \mathit{ff} \}$
as our set of boolean elements. We define an element of $L(\mathbb{B}$
to represent boolean values:
\begin{displaymath}
  \mathrm{Bool} = \{ (\mathit{tt}, \alpha, \synTrue) \mid \alpha \in M \} \cup \{ (\mathit{ff}, \alpha, \synFalse) \mid \alpha \in M \}
\end{displaymath}
Thus, the boolean $\mathit{tt}$ is represented by the value $\synTrue$
and $\mathit{ff}$ is represented by $\synFalse$. In both case, we
allow arbitrary potentials to be attached.

Realisability of the construction and elimination of booleans amounts
to the existence of the following inequalities. In an arbitrary fibre
$L(A)$, we have $I_A \leq \mathit{tt}^* \mathrm{Bool}$ and
$I_A \leq \mathit{ff}^* \mathrm{Bool}$ (treating $\mathit{tt}$ and
$\mathit{ff}$ as constant functions $A \to \mathbb{B}$). These
inequalities are realised by the corresponding $\synTrue/\synFalse$
expression. For conditionals, the types involved are a little more
complex to ensure agreement between boolean manipulations at the
$\Set$-level and the realising computations. To get a realiser for a
condition, we require a set $A$, an element $X \in L(A)$ (standing for
the context) and an element $Y \in L(A \times \mathbb{B})$ (standing
for the target type) and the existence in $L(A)$ of inequalities
$X \leq (\lambda a. (a, \mathit{tt}))^* Y$, for the true case, and
$X \leq (\lambda a. (a, \mathit{ff}))^* Y$, for the false case. When
we have all these, we get in $L(A \times \mathbb{B})$ an inequality
$\pi_1^* X \otimes \pi_2^* \mathrm{Bool} \leq Y$. \bob{back ref to the
  typing rules for non-iterable data types}

\paragraph{Lists} Lists are a little more involved, due to the need to
explicitly manage a context that applies to all elements of the
list. Let $\mathrm{List}(B)$ be the set of lists with elements from a
set $B$. If we have $A : \Set$ and $B : A \to \Set$ and
$X \in L(\Sigma a:A.\, B a)$, then we the resourced lists predicate
$\mathrm{RList}(X) \in L(\Sigma a : A.\, \mathrm{List}(B a))$ must
satisfy the equation:
\begin{displaymath}
  \mathrm{RList}(X) =
  \begin{array}[t]{l}
    \{ ((a, []), \alpha, (\synFalse, *)) \mid \alpha \in M \} \\
    \cup \\
    \{ \begin{array}[t]{@{}l}
         ((a, b :: bs), \alpha, (\synTrue, (v_1, v_2))) \mid \\
         \quad \exists \alpha_1, \alpha_2. 0 \leq M(\alpha, \alpha_1 \oplus \alpha_2) \land ((a, b), \alpha_1, v_1) \in X \land ((a, bs), \alpha_2, v_2) \in \mathrm{RList}(X) \}
       \end{array}
  \end{array}
\end{displaymath}
This equation has a least solution, by induction on the length of the
list being realised. This definition is somewhat involved, but in
essence states that a list is represented by tagged pairs, where
$\synFalse$ represents nil and $\synTrue$ represents cons, and that
the potential is distributed amongst the elements of the list as
needed.

Realisability of the nil and cons constructors, and of pattern
matching on lists is similar to case for booleans. We refer the reader
to the Agda development for the details.

\subsubsection{Realising QTT}

Summing up the previous sections, we have the following key theorem
paving the way to soundness:
\begin{theorem}\label{thm:basic-qtt-realisability-model}
  For every resource monoid with a distinguished sub-monoid
  $(M, M_0)$, we obtain a realisability model of $\mathbb{N}$-QTT with
  $\Pi$ and $\Sigma$ types, a universe, and non-iterable datatypes.

  \bob{State the realisability property of closed programs.}
\end{theorem}

%%%%%%%%%%%%%%%%%%%%%%%%%%%%%%%%%%%%%%%%%%%%%%%%%%%%%%%%%%%%%%%%%%%%%%
%%%%%%%%%%%%%%%%%%%%%%%%%%%%%%%%%%%%%%%%%%%%%%%%%%%%%%%%%%%%%%%%%%%%%%
%%%%%%%%%%%%%%%%%%%%%%%%%%%%%%%%%%%%%%%%%%%%%%%%%%%%%%%%%%%%%%%%%%%%%%
%%%%%%%%%%%%%%%%%%%%%%%%%%%%%%%%%%%%%%%%%%%%%%%%%%%%%%%%%%%%%%%%%%%%%%
\section{Realising Iteration for Implicit Polynomial Time}
\label{sec:realising-iteration}

The models constructed in the previous section only allow for
constant-time programs to be realised.

If we use the natural number resource monoid (\bob{Ref})...

\subsection{Iteration Resource Monoids}

To interpret iteration over a resource monoid $(M, M_0)$, we require
additional structure, which we call an \emph{Iteration Resource
  Monoid} to account for measurement of the sizes of iterable data
structures and the effects of iteration on potentials.

\subsubsection{Definition}
We require:
\begin{enumerate}
\item a function $\mathit{size} : \mathbb{N} \to M$ that gives the
  potential of an iterable data structure of a given
  size;
\item a function $\mathit{raise} : M \to M$ that raises the
  (polynomial) degree of some potential; and
\item a function $\mathit{scale} : \mathbb{N} \times M \to M$ that
  scales a potential for a fixed number of iterations.
\end{enumerate}
These functions must satisfy the following properties:
\begin{enumerate}
\item $M_0$ is closed under the $\mathit{raise}$ operation;
\item for all $\alpha$ and $n$,
  $0 \leq M(\mathit{raise}(\alpha) \oplus \mathit{size}(n),
  \mathit{scale}(n,\alpha) \oplus \mathit{size}(n))$; and
\item for all $\alpha \in M_0$ and $n$,
  $0 \leq M(\mathit{scale}(1 + n, \alpha), \alpha \oplus
  \mathit{scale}(n, \alpha))$.
\end{enumerate}
The first property states that $\mathit{raise}$ is suitable as
potential for whole programs2, meaning that it does not make any
requirements on the existence of iterable data. Note that we do
\emph{not} require $M_0$ to contain $\mathit{size}(n)$ -- programs
themselves may not contain iterable data, all potential for iteration
must be delivered externally. A useful intuition is that
$\mathit{scale}(n,\alpha)$ represents the potential required for at
most $n$ iterations that require potential $\alpha$, whereas
$\mathit{raise}(\alpha)$ represents the potential required for a
number of iterations that depends on the context. This is the
motivation behind the second required property, which states that
having $\mathit{raise}(\alpha)$ potential implies having
$\mathit{scale}(n,\alpha)$ potential when the current size is $n$. The
third property states that $\mathit{scale}$ decomposes as expected on
potentials that do not include any size potential.

Note that $\mathit{scale}(n,\alpha)$ is not the same as the action
$n \cdot \alpha$ defined in \autoref{FIXME}. The latter operation
scales both size and function potential, but the former only scales
the function potential.

\subsubsection{Polynomial Iteration Resource Monoids}

Both of the polymonial resource monoids defined in
\autoref{sec:polynomial-resource-monoids} support the structure of an
Iteration Resource Monoid. We define:
\begin{enumerate}
\item $\mathit{size}(n) = (n, 0)$
\item $\mathit{raise}(n, p) = (n, xp)$
\item $\mathit{scale}(m, (n, p)) = (n, m \cdot p)$
\end{enumerate}
Note that $\mathit{raise}$ does indeed raise the degree of the
polynomial involved. Property 2 above is satisfied because for any
polynomial we have $(m \cdot p)(x) \leq (xp)(x)$ whenever $m \leq x$.

\subsubsection{Realising Iterable Natural Numbers}

For any natural number $n$, we define its representation as a value
$\mathrm{natValue}(n) \in \ValSet$ by recursion.
\begin{displaymath}
  \begin{array}{lcl}
    \mathrm{natValue}(0) &=& (\synTrue , *) \\
    \mathrm{natValue}(1 + n) &=& (\synFalse, \mathrm{natValue}(n))
  \end{array}
\end{displaymath}
This representation uses a tagged pair approach similar to our
representation of lists in \autoref{sec:lists}. Using this, we can
define what it means for a natural number to be realisable via
$\mathrm{Nat} \in L(\mathbb{N})$:
\begin{displaymath}
  \mathrm{Nat} = \{(n, \alpha, \mathrm{natValue}(n)) \mid n \in \mathbb{N}, 0 \leq M(\alpha, \mathit{size}(n)) \}
\end{displaymath}
So a natural number $n$ is realised by the value
$\mathrm{natValue}(n)$ as long as we have at least $\mathit{size}(n)$
potential.  This gives us the ability to represent natural numbers as
a type in QTT, but in order to iterate (and construct in the case of
LFPL), we need to construct specific realisers for the Cons-Free and
LFPL systems.

\subsection{The Cons-Free System}
\label{sec:cons-free-sound}

The Cons-Free system uses the $\MaxPoly$ resource monoid, with the
distinguished sub-monoid being those elements that are $0$ in the size
component.

\begin{enumerate}
\item We can duplicate natural numbers
\item We \emph{cannot} construct natural numbers
\item We can construct a recursor
\end{enumerate}

\begin{theorem}[Soundness for the Cons-free System]
  \label{thm:cons-free-soundness}
  \bob{State the poly-time property for this system}
\end{theorem}

\subsection{The LFPL System}
\label{sec:lfpl-sound}

The LFPL system uses the $\PlusPoly$ resource monoid, with the
distinguished sub-monoid again being those elements that are $0$ in
the size component.

\begin{enumerate}
\item We can no longer duplicate natural numbers
\item Define diamonds
\item We can construct natural numbers given diamonds
\item We can construct a recursor
\end{enumerate}

\begin{theorem}[Soundness for the LFPL-style System]
  \label{thm:lfpl-soundness}
  \bob{State the poly-time property for this system}
\end{theorem}


\subsection{Beyond Natural Numbers}

\bob{Theoretically possible, bit useless in the Cons-Free system,
  useful in the LFPL one}.


%%%%%%%%%%%%%%%%%%%%%%%%%%%%%%%%%%%%%%%%%%%%%%%%%%%%%%%%%%%%%%%%%%%%%%
%%%%%%%%%%%%%%%%%%%%%%%%%%%%%%%%%%%%%%%%%%%%%%%%%%%%%%%%%%%%%%%%%%%%%%
%%%%%%%%%%%%%%%%%%%%%%%%%%%%%%%%%%%%%%%%%%%%%%%%%%%%%%%%%%%%%%%%%%%%%%
%%%%%%%%%%%%%%%%%%%%%%%%%%%%%%%%%%%%%%%%%%%%%%%%%%%%%%%%%%%%%%%%%%%%%%
% \section{Realising Explicit Amortised Complexity}
% \label{sec:explicit-amortised-complexity}

%%%%%%%%%%%%%%%%%%%%%%%%%%%%%%%%%%%%%%%%%%%%%%%%%%%%%%%%%%%%%%%%%%%%%%
%%%%%%%%%%%%%%%%%%%%%%%%%%%%%%%%%%%%%%%%%%%%%%%%%%%%%%%%%%%%%%%%%%%%%%
%%%%%%%%%%%%%%%%%%%%%%%%%%%%%%%%%%%%%%%%%%%%%%%%%%%%%%%%%%%%%%%%%%%%%%
%%%%%%%%%%%%%%%%%%%%%%%%%%%%%%%%%%%%%%%%%%%%%%%%%%%%%%%%%%%%%%%%%%%%%%
\section{Relationship to Second-order MALL with Modalities}
\label{sec:poly-mall}

\begin{enumerate}
\item Compare the SAL and LFPL modalities to the natural number
  constructions
  \begin{enumerate}
  \item In the SAL case, we have $\oc A \multimap A^n$, which allows
    us to
  \end{enumerate}
\end{enumerate}

%%%%%%%%%%%%%%%%%%%%%%%%%%%%%%%%%%%%%%%%%%%%%%%%%%%%%%%%%%%%%%%%%%%%%%
%%%%%%%%%%%%%%%%%%%%%%%%%%%%%%%%%%%%%%%%%%%%%%%%%%%%%%%%%%%%%%%%%%%%%%
%%%%%%%%%%%%%%%%%%%%%%%%%%%%%%%%%%%%%%%%%%%%%%%%%%%%%%%%%%%%%%%%%%%%%%
%%%%%%%%%%%%%%%%%%%%%%%%%%%%%%%%%%%%%%%%%%%%%%%%%%%%%%%%%%%%%%%%%%%%%%
\section{Conclusion}
\label{sec:conclusion}

\begin{enumerate}
\item Brunel's classical quantitative realisability
\item Works on polytime realisability for proof extraction (Schwictenberg, Soft Set Theory)
\item Lower complexity classes, like LOGSPACE
\item Higher complexity classes, like EXPTIME, ELEMENTARY
\item More fine-granied classes, like PTIME($k$)
\item Automation, in the style of Jan's stuff
\item Circuits and shit.
\end{enumerate}


%%
%% The acknowledgments section is defined using the "acks" environment
%% (and NOT an unnumbered section). This ensures the proper
%% identification of the section in the article metadata, and the
%% consistent spelling of the heading.
\begin{acks}
  FIXME
\end{acks}

%%
%% The next two lines define the bibliography style to be used, and
%% the bibliography file.
\bibliographystyle{ACM-Reference-Format}
\bibliography{sample-base}


\end{document}
\endinput
