\appendix

\section{Typing Rules for QTT/Cons-free and QTT/LFPL}
\label{sec:rules}

The judgements of Quantitative Type Theory are as follows:
\begin{mathpar}
  \begin{array}{ll}
    \Gamma~\isctxt&\textrm{contexts} \\
%    \Gamma_1 \equiv \Gamma_2~\isctxt&\textrm{equal contexts} \\
    \Gamma \vdash S~\istype&\textrm{types} \\
    \Gamma \vdash S \equiv T~\istype&\textrm{equal types} \\
  \end{array}

  \begin{array}{ll}
    \Gamma \vdash M \stackrel\sigma: S&\textrm{terms} \\
    \Gamma \vdash M \equiv N \stackrel\sigma: S&\textrm{equal terms}
  \end{array}
\end{mathpar}
In the term and term equality judgements, the usage $\sigma$ is either
$0$ or $1$. We use $\rho$ and $\pi$ to range over arbitrary usages
from the semiring $R$.

\subsection{Context formation}
\begin{mathpar}
  \inferrule* [right=Emp]
  { }
  {\diamond~\isctxt}

  \inferrule* [right=Ext]
  {\Gamma~\isctxt \\ 0\Gamma \vdash S}
  {\Gamma, x \stackrel\rho: S~\isctxt}
\end{mathpar}

\subsection{Type Equality}
\begin{mathpar}
  \inferrule* [right=Ty-Eq-Refl]
  {0\Gamma \vdash S}
  {0\Gamma \vdash S \equiv S}

  \inferrule* [right=Ty-Eq-Symm]
  {0\Gamma \vdash S \equiv T}
  {0\Gamma \vdash T \equiv S}

  \inferrule* [right=Ty-Eq-Tran]
  {0\Gamma \vdash S \equiv T \\ 0\Gamma \vdash T \equiv U}
  {0\Gamma \vdash S \equiv U}
\end{mathpar}
as well as congruence rules for each type formation rule (elided), and
the universe eliminator.

\subsection{Term Equality}
\begin{mathpar}
  \inferrule* [right=Tm-Eq-Refl]
  {\Gamma \vdash M \stackrel\sigma: S}
  {\Gamma \vdash M \equiv M \stackrel\sigma: S}

  \inferrule* [right=Tm-Eq-Symm]
  {\Gamma \vdash M \equiv N \stackrel\sigma: S}
  {\Gamma \vdash N \equiv M \stackrel\sigma: S}

  \inferrule* [right=Tm-Eq-Tran]
  {\Gamma \vdash M \equiv N \stackrel\sigma: S \\
    \Gamma \vdash N \equiv O \stackrel\sigma: S}
  {\Gamma \vdash M \equiv O \stackrel\sigma: S}
\end{mathpar}
as well as congruence rules for each term formation rule (elided), and
the specific $\beta\eta$-equalities for each type listed below.

\subsection{Variables, conversion, sub-usaging}
\begin{mathpar}
  \inferrule* [right=Var]
  {0\Gamma, x \stackrel\sigma: S, 0\Gamma'~\isctxt}
  {0\Gamma, x \stackrel\sigma: S, 0\Gamma' \vdash x \stackrel\sigma: S}

  \inferrule* [right=Conv]
  {\Gamma \vdash M \stackrel\sigma: S \\ 0\Gamma \vdash S \equiv T~\istype}
  {\Gamma \vdash M \stackrel\sigma: T}

  \inferrule* [right=Sub]
  {\Gamma \vdash M \stackrel\sigma: S \\ \Gamma' \sqsubseteq \Gamma}
  {\Gamma' \vdash M \stackrel\sigma: S}
\end{mathpar}

\subsection{$\Pi$-types} Type formation:
\begin{displaymath}
  \inferrule* [right=Ty-Pi]
  {0\Gamma \vdash S \\ 0\Gamma, x \stackrel0: S \vdash T}
  {0\Gamma \vdash (x \stackrel\pi: S) \to T}
\end{displaymath}
Introduction and elimination:
\begin{mathpar}
  \inferrule* [right=Tm-Lam]
  {\Gamma, x \stackrel{\sigma\pi}: S \vdash M \stackrel\sigma: T}
  {\Gamma \vdash \lambda x.M \stackrel\sigma: (x \stackrel\pi: S) \to T}

  \inferrule* [right=Tm-App]
  {\Gamma_1 \vdash M \stackrel\sigma: (x \stackrel\pi: S) \to T \\
    \Gamma_2 \vdash N \stackrel{\sigma'}: S \\
    0\Gamma_1 = 0\Gamma_2 \\
    \sigma' = 0 \Leftrightarrow (\pi = 0 \lor \sigma = 0)
  }
  {\Gamma_1 + \pi\Gamma_2 \vdash M\,N \stackrel\sigma: T[N/x]}
\end{mathpar}
$\beta\eta$-equalities (as well as congruences):
\begin{mathpar}
  \inferrule* [right=Tm-Eq-Pi$\beta$]
  {\Gamma_1, x \stackrel{\sigma\pi}: S \vdash M \stackrel\sigma: T \\
    \Gamma_2 \vdash N \stackrel{\sigma'}: S \\
    (\sigma' = 0 \Leftrightarrow \pi = 0 \lor \sigma = 0)}
  {\Gamma_1 + \pi\Gamma_2 \vdash (\lambda x.M)\, N \equiv M[N/x] \stackrel\sigma: T[N/x]}

  \inferrule* [right=Tm-Eq-Pi$\eta$]
  {\Gamma \vdash M \stackrel\sigma: (x \stackrel\pi: S) \to T}
  {\Gamma \vdash \lambda x .M\,x \equiv M \stackrel\sigma: (x \stackrel\pi: S) \to T}
\end{mathpar}

\subsection{$\Sigma$-types}
Type formation:
\begin{mathpar}
  \inferrule* [right=Ty-Tensor]
  {0\Gamma \vdash S~\istype \\ 0\Gamma, x \stackrel0: S \vdash T~\istype}
  {0\Gamma \vdash (x \stackrel\pi: S) \otimes T~\istype}

  \inferrule* [right=Ty-Unit]
  {0\Gamma~\isctxt}
  {0\Gamma \vdash I~\istype}
\end{mathpar}
Introduction:
\begin{mathpar}
  \inferrule* [right=Tm-Pair]
  {\Gamma_1 \vdash M \stackrel{\sigma'}: S \\
    \Gamma_2 \vdash N \stackrel\sigma: T[M/x] \\
    0\Gamma_1 = 0\Gamma_2 \\
    \sigma' = 0 \Leftrightarrow (\pi = 0 \lor \sigma = 0)}
  {\pi\Gamma_1 + \Gamma_2 \vdash (M, N) \stackrel\sigma: (x \stackrel\pi: S) \otimes T}

  \inferrule* [right=Tm-Unit]
  {0\Gamma~\isctxt}
  {0\Gamma \vdash * \stackrel\sigma: I}
\end{mathpar}
$\sigma = 0$ fragment eliminators:
\begin{mathpar}
  \inferrule* [right=Tm-Fst]
  {\Gamma \vdash M \stackrel0: (x \stackrel\pi: S) \otimes T}
  {\Gamma \vdash \mathrm{fst}(M) \stackrel0: S}

  \inferrule* [right=Tm-Snd]
  {\Gamma \vdash M \stackrel0: (x \stackrel\pi: S) \otimes T}
  {\Gamma \vdash \mathrm{snd}(M) \stackrel0: T[\mathrm{fst}(M)/x]}
\end{mathpar}
Linear eliminators:
\begin{mathpar}
  \inferrule* [right=Tm-Let-Pair]
  {0\Gamma, z \stackrel0: (x \stackrel\pi: A) \otimes B \vdash C~\istype \\
    \Gamma_1 \vdash M \stackrel\sigma: (x \stackrel\pi: A) \otimes B \\
    \Gamma_2, x \stackrel{\sigma\pi}: A, y \stackrel{\sigma}: B \vdash N \stackrel\sigma: C[(x,y)/z]\\
    0\Gamma_1 = 0\Gamma_2}
  {\Gamma_1 + \Gamma_2 \vdash \mathrm{let}~(x,y) = M~\mathrm{in}~N \stackrel\sigma: C[M/z]}

  \inferrule* [right=Tm-Let-Unit]
  {0\Gamma_1, x \stackrel0: I \vdash C~\istype \\
    \Gamma_1 \vdash M \stackrel\sigma: I \\
    \Gamma_2 \vdash N \stackrel\sigma: C[*/x] \\
    0\Gamma_1 = 0\Gamma_2}
  {\Gamma_1+\Gamma_2 \vdash \mathrm{let}~* = M~\mathrm{in}~N \stackrel\sigma: C[M/x]}
\end{mathpar}
$\beta$-equality for both fragments:
\begin{mathpar}
  \inferrule*
  {0\Gamma, z \stackrel0: (x \stackrel\pi: A) \otimes B \vdash C~\istype \\
    \Gamma_1 \vdash M_1 \stackrel{\sigma'}: A \\
    \Gamma_2 \vdash M_2 \stackrel\sigma: B[M/x] \\
    \sigma' = 0 \Leftrightarrow (\pi = 0 \lor \sigma = 0) \\
    \Gamma_3, x \stackrel{\sigma\pi}: A, y \stackrel{\sigma}: B \vdash N \stackrel\sigma: C[(x,y)/z]\\
    0\Gamma_1 = 0\Gamma_2 = 0\Gamma_3}
  {\pi\Gamma_1 + \Gamma_2 + \Gamma_3 \vdash \mathrm{let}~(x,y) = (M_1,M_2)~\mathrm{in}~N \equiv N[M_1/x,M_2/y]\stackrel\sigma: C[(M_1,M_2)/z]}

  \inferrule*
  {0\Gamma, x \stackrel0: I \vdash C~\istype \\
    \Gamma \vdash N \stackrel\sigma: C[*/x]}
  {\Gamma \vdash \mathrm{let}~* = *~\mathrm{in}~N \equiv N[*/x] \stackrel\sigma: C[*/x]}
\end{mathpar}
$\beta\eta$-equalities for the $\sigma = 0$ fragment:
\begin{mathpar}
  \inferrule* [right=Eq-Pair-Fst]
  {0\Gamma \vdash M \stackrel0: S \\
    0\Gamma \vdash N \stackrel0: T[M/x]}
  {0\Gamma \vdash \mathrm{fst}(M, N) \equiv M \stackrel0: S}

  \inferrule* [right=Eq-Pair-Snd]
  {0\Gamma \vdash M \stackrel0: S \\
    0\Gamma \vdash N \stackrel0: T[M/x]}
  {0\Gamma \vdash \mathrm{snd}(M, N) \equiv N \stackrel0: T[M/x]}

  \inferrule* [right=Eq-Unit-$\eta$]
  {0\Gamma \vdash M \stackrel0: I}
  {0\Gamma \vdash M \equiv * \stackrel0: I}

  \inferrule* [right=Eq-Pair-$\eta$]
  {0\Gamma \vdash M \stackrel0: (x \stackrel\pi: S) \otimes T}
  {0\Gamma \vdash (\mathrm{fst}(M), \mathrm{snd}(M)) \equiv M \stackrel0: (x \stackrel\pi: S) \otimes T}

  \inferrule*
  {0\Gamma, z \stackrel0: (x \stackrel\pi: A) \otimes B \vdash C \\
    0\Gamma \vdash M \stackrel0: (x \stackrel\pi: A) \otimes B \\
    0\Gamma, x \stackrel0: A, y \stackrel0: B \vdash N \stackrel0: C[(x,y)/z]}
  {0\Gamma \vdash \mathrm{let}~(x,y) = M~\mathrm{in}~N \equiv N[\mathrm{fst}(M)/x,\mathrm{snd}(M)/y] \stackrel0: C[M/z]}
\end{mathpar}

\subsection{Identity Type} Type formation, introduction, reflection
and $\eta$-law:
\begin{mathpar}
  \inferrule* [right=Ty-Id]
  {0\Gamma \vdash S~\istype \\
    0\Gamma \vdash M \stackrel0: S \\
    0\Gamma \vdash N \stackrel0: S}
  {0\Gamma \vdash M =_S N~\istype}

  \inferrule* [right=Id-Refl]
  {\Gamma \vdash M \stackrel\sigma: S}
  {\Gamma \vdash \conRefl(M) \stackrel\sigma: M =_S M}

  \inferrule* [right=Id-Reflect]
  {\Gamma \vdash N \stackrel0: M_1 =_S M_2}
  {\Gamma \vdash M_1 \equiv M_2 \stackrel0: S}

  \inferrule* [right=Id-Uniq]
  {0\Gamma \vdash P \stackrel0: M =_A M}
  {0\Gamma \vdash P \equiv \mathrm{refl}(M) : M=_A M}
\end{mathpar}

\subsection{Universe}

\newcommand{\TySet}{\mathsf{U}}
\newcommand{\TyEl}{\mathsf{El}}
Type formation:
\begin{displaymath}
  \inferrule* [right=Ty-U]
  {0\Gamma~\isctxt}
  {0\Gamma \vdash \TySet~\istype}
\end{displaymath}
Introduction (also with introduction rules for all other type formers except $\TySet$):
\begin{mathpar}
  \inferrule* [right=Tm-U-Bool]
  {\Gamma~\isctxt}
  {\Gamma \vdash \BoolTy \stackrel\sigma: \TySet}

  \inferrule* [right=Tm-U-Tensor]
  {\Gamma \vdash M \stackrel\sigma: \TySet \\
    \Gamma, x \stackrel\rho: \TyEl(M) \vdash N \stackrel\sigma: \TySet}
  {\Gamma \vdash (x \stackrel\pi: M) \otimes N \stackrel\sigma: \TySet}

  \ldots
\end{mathpar}
Elimination:
\begin{displaymath}
  \inferrule* [right=Ty-El]
  {0\Gamma \vdash M \stackrel0: \TySet}
  {0\Gamma \vdash \TyEl(M)~\istype}
\end{displaymath}
Equality:
\begin{mathpar}
  % \inferrule* [right=Ty-Eq-El-Pi]
  % {0\Gamma \vdash M \stackrel0: \TySet \\
  %   0\Gamma, x \stackrel0: \TyEl(M) \vdash N \stackrel0: \TySet}
  % {0\Gamma \vdash \TyEl((x \stackrel\rho: M) \to N) \equiv (x \stackrel\rho: \TyEl(M)) \to \TyEl(N)~\istype}

  % \inferrule* [right=Ty-Eq-El-Bool]
  % {0\Gamma~\isctxt}
  % {0\Gamma \vdash \TyEl(\BoolTy) \equiv \BoolTy~\istype}

  % \ldots

  \inferrule* [right=Ty-Eq-El-Cong]
  {0\Gamma \vdash M \equiv N \stackrel0: \TySet}
  {0\Gamma \vdash \TyEl(M) \equiv \TyEl(N)~\istype}
\end{mathpar}

\subsection{Booleans}
Formation, introduction, and elimination:
\begin{mathpar}
  \inferrule*
  {\Gamma~\isctxt}
  {\Gamma \vdash \BoolTy~\istype}

  \inferrule*
  {\Gamma~\isctxt}
  {0\Gamma \vdash \cTrue, \cFalse \stackrel\sigma: \BoolTy}

  \inferrule*
  {0\Gamma_1, x \stackrel0: \BoolTy \vdash P~\istype \\
    \Gamma_1 \vdash M \stackrel\sigma: \BoolTy \\
    \Gamma_2 \vdash N_t \stackrel\sigma: P[\cTrue/x] \\
    \Gamma_2 \vdash N_f \stackrel\sigma: P[\cFalse/x] \\
    0\Gamma_1 = 0\Gamma_2}
  {\Gamma_1 + \Gamma_2 \vdash \If_{x.P}\: M \: \Then \: N_t \: \Else \: N_f \stackrel\sigma: P[M/x]}
\end{mathpar}
$\beta$-equalities:
\begin{mathpar}
  \inferrule* [right=Tm-Eq-True$\beta$]
  {0\Gamma, z \stackrel0: \BoolTy \vdash P \\
    \Gamma \vdash M_t \stackrel\sigma: P[\mathrm{true}/z] \\
    \Gamma \vdash M_f \stackrel\sigma: P[\mathrm{false}/z]}
  {\Gamma \vdash \If_{x.P}\: \cTrue \: \Then \: N_t \: \Else \: N_f \equiv N_t \stackrel\sigma: P[\mathrm{true}/z]}

  \inferrule* [right=Tm-Eq-False$\beta$]
  {0\Gamma, z \stackrel0: \BoolTy \vdash P \\
    \Gamma \vdash M_t \stackrel\sigma: P[\mathrm{true}/z] \\
    \Gamma \vdash M_f \stackrel\sigma: P[\mathrm{false}/z]}
  {\Gamma \vdash \If_{x.P}\: \cFalse \: \Then \: N_t \: \Else \: N_f \equiv N_f \stackrel\sigma: P[\mathrm{false}/z]}
\end{mathpar}

\subsection{Lists} Formation and introduction:
\begin{mathpar}
  \inferrule*
  {0\Gamma \vdash T~\istype}
  {0\Gamma \vdash \ListTy(T)~\istype}

  \inferrule*
  {\Gamma \vdash T~\istype}
  {0\Gamma \vdash \cNil \stackrel\sigma: \ListTy(T)}

  \inferrule*
  {\Gamma_1 \vdash M \stackrel\sigma: T \\
    \Gamma_2 \vdash N \stackrel\sigma: \ListTy(T) \\
    0\Gamma_1 = 0\Gamma_2}
  {\Gamma_1 + \Gamma_2 \vdash \cCons(M,N) \stackrel\sigma: \ListTy(T)}
\end{mathpar}
Case analysis:
\begin{mathpar}
  \inferrule*
  {0\Gamma_1, x \stackrel0: \ListTy(T) \vdash P~\istype \\
    \Gamma_1 \vdash M \stackrel\sigma: \ListTy(T) \\
    \Gamma_2 \vdash N_1 \stackrel\sigma: P[\cNil/x] \\
    \Gamma_2, h \stackrel\sigma: T, t \stackrel\sigma: \ListTy(T) \vdash N_2 \stackrel\sigma: P[\cCons(h,t)/x] \\
    0\Gamma_1 = 0\Gamma_2}
  {\Gamma_1 + \Gamma_2 \vdash \Match_{x.P}\,M\,\{\,\cNil \mapsto N_1; \cCons(h,t) \mapsto N_2\,\} \stackrel\sigma: P[M/x]}
\end{mathpar}
$\beta$-equalities for case analysis:
\begin{mathpar}
  \inferrule* [right=Eq-List-Match-Nil]
  {0\Gamma, x : \stackrel0: \ListTy(T) \vdash P~\istype \\
    \Gamma \vdash N_1 \stackrel\sigma: P[\cNil/x] \\
    \Gamma, h \stackrel\sigma: T, t \stackrel\sigma: \ListTy(T) \vdash N_2 \stackrel\sigma: P[\cCons(h,t)/x]}
  {\Gamma \vdash \Match_{x.P}\,\cNil\,\{\,\cNil \mapsto N_1; \cCons(h,t) \mapsto N_2\,\} \equiv N_1 \stackrel\sigma: P[\cNil/x]}

  \inferrule* [right=Eq-List-Match-Cons]
  {0\Gamma_1, x : \stackrel0: \ListTy(T) \vdash P~\istype \\
    \Gamma_1 \vdash M_1 \stackrel\sigma: T \\
    \Gamma_2 \vdash M_2 \stackrel\sigma: \ListTy(T) \\
    \Gamma_3 \vdash N_1 \stackrel\sigma: P[\cNil/x] \\
    \Gamma_3, h \stackrel\sigma: T, t \stackrel\sigma: \ListTy(T) \vdash N_2 \stackrel\sigma: P[\cCons(h,t)/x] \\
    0\Gamma_1 = 0\Gamma_2 = 0\Gamma_3}
  {\Gamma_1 + \Gamma_2 + \Gamma_3 \vdash {\begin{array}[m]{@{}l}
                                            \Match_{x.P}\,(\cCons(M_1,M_2))\\
                                            \qquad\{\,\cNil \mapsto N_1; \cCons(h,t) \mapsto N_2\,\}\\
                                            \equiv N_2[M_1/h,M_2/t] \end{array}} \stackrel\sigma: P[\cCons(M_1,M_2)/x]}
\end{mathpar}
$\sigma = 0$ recursive eliminator:
\begin{mathpar}
  \inferrule*
  {0\Gamma, x \stackrel0: \ListTy(T) \vdash P~\istype \\
    0\Gamma \vdash M \stackrel0: \ListTy(T) \\
    0\Gamma \vdash N_1 \stackrel0: P[\cNil/x] \\
    0\Gamma, h \stackrel0: T, t \stackrel0: \ListTy(T), p \stackrel0: P[t/x] \vdash N_2 \stackrel0: P[\cCons(h,t)/x]}
  {0\Gamma \vdash \mathrm{recList}_{x.P}\,M\,\{\,\cNil \mapsto N_1; \cCons(h,t;p) \mapsto N_2\,\} \stackrel0: P[M/x]}
\end{mathpar}
$\beta$-equalities for recursion:
\begin{mathpar}
  \inferrule* [right=Eq-List-Rec-Nil]
  {0\Gamma, x : \stackrel0: \ListTy(T) \vdash P~\istype \\
    0\Gamma \vdash N_1 \stackrel0: P[\cNil/x] \\
    0\Gamma, h \stackrel\sigma: T, t \stackrel\sigma: \ListTy(T) \vdash N_2 \stackrel0: P[\cCons(h,t)/x]}
  {0\Gamma \vdash \mathrm{recList}_{x.P}\,\cNil\,\{\,\cNil \mapsto N_1; \cCons(h,t;p) \mapsto N_2\,\} \equiv N_1 \stackrel0: P[\cNil/x]}

  \inferrule* [right=Eq-List-Rec-Cons]
  {0\Gamma, x : \stackrel0: \ListTy(T) \vdash P~\istype \\
    0\Gamma \vdash M_1 \stackrel0: T \\
    0\Gamma \vdash M_2 \stackrel0: \ListTy(T) \\
    0\Gamma \vdash N_1 \stackrel0: P[\cNil/x] \\
    0\Gamma, h \stackrel0: T, t \stackrel0: \ListTy(T) \vdash N_2 \stackrel0: P[\cCons(h,t)/x]}
  {0\Gamma \vdash {\begin{array}[m]{@{}l}
                     \mathrm{recList}_{x.P}\,(\cCons(M_1,M_2))\,\{\,\cNil \mapsto N_1; \cCons(h,t;p) \mapsto N_2\,\}\\
                     \equiv N_2[\begin{array}[t]{@{}l} M_1/h,M_2/t,\\
                                  \mathrm{recList}_{x,p}\,M_2\,\{\cNil \mapsto N_1; \cCons(h,t;p) \mapsto N_2\,\}/p]\end{array} \end{array}} \stackrel0: P[\cCons(M_1,M_2)/x]}
\end{mathpar}

\subsection{Cons-Free Naturals} Formation and introduction:
\begin{mathpar}
  \inferrule*
  {0\Gamma~\isctxt}
  {0\Gamma \vdash \tyNat~\istype}

  \inferrule*
  {\Gamma~\isctxt}
  {\Gamma \vdash \conZero \stackrel0: \tyNat}

  \inferrule*
  {\Gamma \vdash M \stackrel0: \tyNat}
  {\Gamma \vdash \conSucc(M) \stackrel0: \tyNat}
\end{mathpar}
Duplication:
\begin{mathpar}
  \inferrule*
  {\Gamma \vdash M \stackrel\sigma: \tyNat}
  {\Gamma \vdash \mathrm{dupNat}(M) \stackrel\sigma: \tyNat \otimes \tyNat}

  \inferrule*
  {\Gamma \vdash M \stackrel0: \tyNat}
  {\Gamma \vdash \mathrm{dupNat}(M) \equiv (M,M) \stackrel0: \tyNat \otimes \tyNat}
\end{mathpar}
Eliminator:
\begin{displaymath}
  \mprset{flushleft}
  \inferrule*
  {0\Gamma, x \stackrel0: \tyNat \vdash P~\istype \\\\
    \Gamma \vdash M \stackrel\sigma: \tyNat \\\\
    0\Gamma \vdash N_z \stackrel\sigma: P[\conZero/x] \\\\
    0\Gamma, n \stackrel0: \tyNat, p \stackrel\sigma: P[n/x] \vdash N_s \stackrel\sigma: P[\conSucc(n)/x]}
  {\Gamma \vdash \tmRec_{x.P}\,M\,\{\conZero \mapsto N_z; \conSucc(n;p) \mapsto N_s\} \stackrel\sigma: P[M/x]}
\end{displaymath}
$\beta$-equalties (only available in the $\sigma = 0$ fragment):
\begin{mathpar}
  \mprset{flushleft}
  \inferrule*
  {0\Gamma, x \stackrel0: \tyNat \vdash P~\istype \\\\
    0\Gamma \vdash N_z \stackrel0: P[\conZero/x] \\\\
    0\Gamma, n \stackrel0: \tyNat, p \stackrel0: P[n/x] \vdash N_s \stackrel0: P[\conSucc(n)/x]}
  {0\Gamma \vdash \tmRec_{x.P}\,\conZero\,\{\conZero \mapsto N_z; \conSucc(n;p) \mapsto N_s\} \equiv N_z \stackrel0: P[\conZero/x]}

  \mprset{flushleft}
  \inferrule*
  {0\Gamma, x \stackrel0: \tyNat \vdash P~\istype \\\\
    0\Gamma \vdash M \stackrel0: \tyNat \\\\
    0\Gamma \vdash N_z \stackrel0: P[\conZero/x] \\\\
    0\Gamma, n \stackrel0: \tyNat, p \stackrel0: P[n/x] \vdash N_s \stackrel0: P[\conSucc(n)/x]}
  {0\Gamma \vdash {\begin{array}{@{}l}
                    \tmRec_{x.P}\,(\conSucc(M))\,\{\conZero \mapsto N_z; \conSucc(n;p) \mapsto N_s\} \\
                    \equiv N_s[M/n,\tmRec_{x.P}\,M\,\{\conZero \mapsto N_z; \conSucc(n;p) \mapsto N_s\}/p] \end{array}} \stackrel0: P[\conSucc(M)/x]}
\end{mathpar}

\subsection{LFPL Diamonds} Formation, introduction, and $\eta$-law:
\begin{mathpar}
  \inferrule*
  {\Gamma~\isctxt}
  {0\Gamma \vdash \Diamond~\istype}

  \inferrule*
  {\Gamma~\isctxt}
  {0\Gamma \vdash * \stackrel0: \Diamond}

  \inferrule*
  {\Gamma \vdash M \stackrel0: \Diamond}
  {\Gamma \vdash M \equiv * \stackrel0: \Diamond}
\end{mathpar}

\subsection{LFPL Naturals} Type formation and introduction:
\begin{mathpar}
  \inferrule*
  {0\Gamma~\isctxt}
  {0\Gamma \vdash \tyNat~\istype}

  \inferrule*
  {\Gamma \vdash M \stackrel\sigma: \Diamond}
  {\Gamma \vdash \conZero(M) \stackrel\sigma: \tyNat}

  \inferrule*
  {\Gamma_1 \vdash M \stackrel\sigma: \Diamond \\
    \Gamma_2 \vdash N \stackrel\sigma: \tyNat \\
    0\Gamma_1 = 0\Gamma_2}
  {\Gamma_1 + \Gamma_2 \vdash \conSucc(M,N) \stackrel\sigma: \tyNat}
\end{mathpar}
Elimination:
\begin{mathpar}
  \mprset{flushleft}
  \inferrule*
  {0\Gamma, x \stackrel0: \tyNat \vdash P~\istype \\\\
    \Gamma \vdash M \stackrel\sigma: \tyNat \\\\
    0\Gamma, d \stackrel\sigma: \Diamond \vdash N_z \stackrel\sigma: P[\conZero(*)/x] \\\\
    0\Gamma, d \stackrel\sigma: \Diamond, n \stackrel0: \tyNat, p \stackrel\sigma: P[n/x] \vdash N_s \stackrel\sigma : P[\conSucc(*,n)/x]}
  {\Gamma \vdash \tmRec\,M\,\{\conZero(d) \mapsto N_z; \conSucc(d,n;p) \mapsto N_s\} \stackrel\sigma: P[M/x]}
\end{mathpar}
$\beta$-equalities, available in both fragments:
\begin{mathpar}
  \mprset{flushleft}
  \inferrule*
  {0\Gamma, x \stackrel0: \tyNat \vdash P~\istype \\\\
    \Gamma \vdash M \stackrel\sigma: \Diamond \\\\
    0\Gamma, d \stackrel\sigma: \Diamond \vdash N_z \stackrel\sigma: P[\conZero(*)/x] \\\\
    0\Gamma, d \stackrel\sigma: \Diamond, n \stackrel0: \tyNat, p \stackrel\sigma: P[n/x] \vdash N_s \stackrel\sigma : P[\conSucc(*,n)/x]}
  {\Gamma \vdash \tmRec\,\conZero(M)\,\{\conZero(d) \mapsto N_z; \conSucc(d,n;p) \mapsto N_s\} \equiv N_z[M/d] \stackrel\sigma: P[\conZero(*)/x]}

  \mprset{flushleft}
  \inferrule*
  {0\Gamma, x \stackrel0: \tyNat \vdash P~\istype \\\\
    \Gamma_1 \vdash M_d \stackrel\sigma: \Diamond \\\\
    \Gamma_2 \vdash M_n \stackrel\sigma: \tyNat \\\\
    0\Gamma, d \stackrel\sigma: \Diamond \vdash N_z \stackrel\sigma: P[\conZero(*)/x] \\\\
    0\Gamma, d \stackrel\sigma: \Diamond, n \stackrel0: \tyNat, p \stackrel\sigma: P[n/x] \vdash N_s \stackrel\sigma : P[\conSucc(*,n)/x] \\\\
    0\Gamma_1 = 0\Gamma_2 = 0\Gamma}
  {\Gamma_1 + \Gamma_2 \vdash {\begin{array}{@{}l}
                                 \tmRec\,(\conSucc(M_d,M_n))\,\{\conZero(d) \mapsto N_z; \conSucc(d,n;p) \mapsto N_s\} \\
                                 \equiv N_s[M_d/d, M_n/n, \tmRec\,M_n\,\{\conZero(d) \mapsto N_z; \conSucc(d,n;p) \mapsto N_s\}/p] \end{array}} \stackrel\sigma: P[\conSucc(*,M_n)/x]}
\end{mathpar}

\subsection{Realisability Reflection} Type formation, introduction and
elimination:
\begin{mathpar}
  \inferrule*
  {0\Gamma \vdash A~\istype}
  {0\Gamma \vdash \Rtype(A)~\istype}

  \inferrule*
  {0\Gamma \vdash M \stackrel1: A}
  {0\Gamma \vdash \rIntro(M) \stackrel\sigma: \Rtype(A)}

  \inferrule*
  {\Gamma \vdash M \stackrel\sigma: \Rtype(A)}
  {\Gamma \vdash \rElim(M) \stackrel{\sigma'}: A}
\end{mathpar}
Equalities:
\begin{mathpar}
  \inferrule*
  {0\Gamma \vdash M \stackrel1: A}
  {0\Gamma \vdash \rElim(\rIntro(M)) \equiv M \stackrel\sigma: A}

  \inferrule*
  {0\Gamma \vdash M \stackrel\sigma: \Rtype(A)}
  {0\Gamma \vdash \rIntro(\rElim(M)) \equiv M \stackrel\sigma: A}
\end{mathpar}
