% -*- TeX-engine: xetex -*-

\documentclass[xetex,serif,mathserif,aspectratio=169]{beamer}

\usepackage{mathpartir}
\usepackage{stmaryrd}
\usepackage{attrib}
\usepackage{rotating}
\usepackage{multirow}
\usepackage{colortbl}
\usepackage{mathpartir}
\usepackage{tikz}
\usepackage{cmll}

\usepackage{libertine}
\usepackage{newtxmath}
\usepackage{zi4}

\newcommand{\TyBool}{\mathsf{Bool}}
\newcommand{\TySet}{\mathsf{Set}}
\newcommand{\TyEl}{\mathsf{El}}

\newcommand{\Ty}{\mathrm{Ty}}
\newcommand{\Tm}{\mathrm{Tm}}
\newcommand{\RTm}{\mathrm{RTm}}
\newcommand{\wk}{\mathsf{wk}}
\newcommand{\proj}{\mathsf{p}}
\newcommand{\vartm}{\mathsf{v}}
\newcommand{\sem}[1]{\llbracket #1 \rrbracket}
\newcommand{\cat}[1]{\mathcal{#1}}
\newcommand{\id}{\mathrm{id}}

\newcommand{\op}{\mathsf{op}}
\newcommand{\Set}{\mathrm{Set}}
\newcommand{\skw}[1]{\mathit{#1}}

\usepackage{mathtools}
\DeclarePairedDelimiter{\floor}{\lfloor}{\rfloor}

\newcommand{\tyPrim}[2]{\textup{\texttt{#1}}\langle #2 \rangle}

\newcommand{\sechead}[1]{\textcolor{titlered}{\emph{#1}}}

\newcommand{\typeOfCartSp}[1]{\lbag #1 \rbag}


\def\greyuntil<#1>#2{{\temporal<#1>{\color{black!40}}{\color{black}}{\color{black}} #2}}
\def\greyfrom<#1>#2{{\temporal<#1>{\color{black}}{\color{black!40}}{\color{black!40}} #2}}

\newcommand{\superscript}[1]{\ensuremath{^{\textrm{#1}}}}
\newcommand{\highlight}{\textbf}

\newcommand{\atomprop}{\mathrm}
\newcommand{\true}{\mathbf{T}}
\newcommand{\false}{\mathbf{F}}

\newcommand{\append}{\mathop{+\kern-3pt+}}

\newcommand{\citationgrey}[1]{\textcolor{black!60}{#1}}

\definecolor{titlered}{rgb}{0.8,0.0,0.0}

\newcommand{\hlchange}[1]{\setlength{\fboxsep}{1pt}\colorbox{black!20}{$#1$}}
\newcommand{\altdiff}[3]{\alt<-#1>{#2}{\hlchange{#3}}}

%%%%%%%%%%%%%%%%%%%%%%%%%%%%%%%%%%%%%%%%%%%%%%%%%%%%%%%%%%%%%%%%%%%%%%%%%%%%%%%%
% from http://tex.stackexchange.com/questions/118410/highlight-terms-in-equation-mode
\newlength{\overwritelength}
\newlength{\minimumoverwritelength}
\setlength{\minimumoverwritelength}{0.1cm}
\def\overwrite<#1>#2#3{%
  \settowidth{\overwritelength}{$#2$}%
  \ifdim\overwritelength<\minimumoverwritelength%
    \setlength{\overwritelength}{\minimumoverwritelength}\fi%
  \temporal<#1>{#2}%
    {\stackrel%
      {\begin{minipage}{\overwritelength}%
          \color{red}\centering\small #3\\%
          \rule{1pt}{9pt}%
        \end{minipage}}%
      {\colorbox{red!50}{\color{black}$\displaystyle#2$}}}%
    {\stackrel%
      {\begin{minipage}{\overwritelength}%
          \color{red}\centering\small #3\\%
          \rule{1pt}{9pt}%
        \end{minipage}}%
      {\colorbox{red!50}{\color{black}$\displaystyle#2$}}}}
%%%%%%%%%%%%%%%%%%%%%%%%%%%%%%%%%%%%%%%%%%%%%%%%%%%%%%%%%%%%%%%%%%%%%%%%%%%%%

\setbeamertemplate{navigation symbols}{}
\usecolortheme[rgb={0.8,0,0}]{structure}
\usefonttheme{serif}
\usefonttheme{structurebold}
\setbeamercolor{description item}{fg=black}

\title{Unifying Models of Linear/Graded Dependent Types}
\author{Robert Atkey \\
  \emph{University of Strathclyde} \\
  robert.atkey@strath.ac.uk}
\date{Graded Types Meeting, U. of Kent, 17th June 2022}

\newcommand{\youtem}{\quad \textcolor{titlered!80}{---} \quad}

\newcommand{\titlecard}[1]{\begin{frame}%
  \begin{center}%
    \Large \textcolor{titlered}{#1}%
  \end{center}%
\end{frame}}

\newcommand{\HEAD}[1]{\textcolor{titlered}{#1}}

\begin{document}

\frame{\titlepage}

\begin{frame}
  Lots of proposals for combining type dependency with
  gradings/linearity/intensional information.

  \bigskip

  \begin{itemize}
  \item Linear Dependent Types \citationgrey{(V{\'a}k{\'a}r, 2014)}
  \item Integrating Linear and Dependent Types \citationgrey{(Krishnaswami, Pradic, Benton, 2015)}
  \item Quantitative Type Theory \citationgrey{(McBride, 2016; Atkey, 2018)}
  \item Linear DTT for Quantum Programming Languages \citationgrey{(Fu, Kishida, Selinger, 2020)}
  \item Graded Modal Dependent Type Theory \citationgrey{(Moon, Eades, Orchard, 2021)}
  \item Graded Dependent Type System \citationgrey{(Choudhury, Eades, Eisenberg, Weirich, 2021)}
  \item Parametric Quantifiers \citationgrey{(Nuyts, Vezzosi, Devriese, 2017)}
  \item Multi-modal type theory \citationgrey{(Gratzer, Kaavos, Nuyts, Birkedal, 2021)}
  \item Dependent Dependency Calculus \citationgrey{(Choudhury, Eades, Weirich, 2022)}
  \item ...
  \end{itemize}
\end{frame}

\titlecard{But what is \emph{really} going on?}

\begin{frame}

  \begin{enumerate}
  \item<1-> Linearity vs. dependency. \\
    {\small Does mentioning a variable in a type count as a use? Can
      specifications consume resources?}

    \bigskip
  \item<2-> Intensional information used for specification \\
    {\small productivity; information flow; parametricity; ...}

    \bigskip
  \item<3-> Intensional information used for runtime properties \\
    {\small resource usage; “sub-Turing” implementability}
  \end{enumerate}

\end{frame}

\begin{frame}
  \frametitle{Linearity vs Dependency}

  QTT pretends that you can interleave linearity and dependency
  \begin{displaymath}
    x_1 \stackrel{\rho_1}: A_1, ..., x_n \stackrel{\rho_n}: A_n \vdash M \stackrel\sigma: B
  \end{displaymath}
  \begin{itemize}
  \item But really there are two systems; $\sigma = 0$ and $\sigma = 1$
  \item Every type has two parts:
    \begin{itemize}
    \item a specification part (normal type theory)
    \item an implementation part (linear type theory)
    \end{itemize}
  \item In the $\sigma = 0$ fragment, only the specification part
    matters; in particular types can only depend on the specification
    part.
  \end{itemize}
\end{frame}

\begin{frame}
  \frametitle{Linearity vs Dependency}

  Split context systems \citationgrey{(Krishnaswami et al., 2015; V{\'a}k{\'a}r, 2014)}
  \begin{displaymath}
    x_1 : A_1, ..., x_n : A_n \mid y_1 : X_1, ..., y_m : X_m \vdash M : Y
  \end{displaymath}

  \bigskip
  Explicit separation of specification types $A_i$ and implementation types $X_j$.

  \bigskip
  Conjectures:
  \begin{enumerate}
  \item QTT can be translated into a split context system
  \item Split context system can be translated into QTT \\
    \quad {\small (probably; terms and conditions apply)}
  \item GMDTT and GraD can be translated into split context system
    (assuming a universe)
  \end{enumerate}

  \pause
  \begin{center}
    {\Large All Linear Dependent Type Theories are a mode of use
      of split contexts}
  \end{center}
\end{frame}

\begin{frame}
  \frametitle{General Semantic Picture}

  Looking at models helps us (me) understand how these interact.

  \bigskip

  \citationgrey{(V{\'a}k{\'a}r, 2015)}
  \begin{displaymath}
    L : \mathcal{C}^{op} \to \textbf{SMCCat}
  \end{displaymath}

  \emph{i.e.}, $\mathcal{C}$-indexed symmetric monoidal categories,
  plus $\mathcal{C}$ is a \emph{Category with Families} and:
  \begin{displaymath}
    L_{\Gamma.A}(\mathsf{p}^*X,Y) \cong L_\Gamma(X, \forall_AY)
  \end{displaymath}

  \medskip
  \begin{itemize}
  \item QTT: Grothendieck construction applied to $L$ \\
    {\small $\cdot$ \quad but each $L_\Gamma$ must be thin! (can this be relaxed?)} \\
    {\small $\cdot$ \quad need a theory of graded multicategories}
  \item QTT is essentially a form of abstract realisability
  \item The $\mathsf{G}$ constructor of Krishnaswami et al.~yields a
    notion of ``reflection'' for QTT
  \item Could also be expressed in terms of \emph{monoidal fibrations}
    or \emph{internal categories}.
  \end{itemize}

\end{frame}

% \begin{frame}
%   \frametitle{Reflection}

%   Starting on making the implementation and specification parts interact:

%   \begin{displaymath}
%     \mathrm{Tm}(\Gamma, GX) \cong L_\Gamma(I, X)
%   \end{displaymath}

%   \begin{mathpar}
%     \inferrule*
%     {0\Gamma \vdash M \stackrel{1}: X}
%     {0\Gamma \vdash \mathsf{G}M \stackrel{0}: \mathsf{G}X}

%     \inferrule*
%     {\Gamma \vdash M \stackrel{\sigma}: \mathsf{G}X}
%     {\Gamma \vdash \mathsf{G}^{-1}M \stackrel{1}: X}
%   \end{mathpar}

%   \begin{itemize}
%   \item
%   \end{itemize}
% \end{frame}

\begin{frame}

  \begin{enumerate}
  \item Do these translations or this general model actually work?

    \bigskip
  \item Can we reason about the intensional information in
    implementations?

    \medskip {\small e.g., if we know a term is polytime computable,
      then what can we do with this?}

    \bigskip
  \item How can we systematically create systems that combine
    intensional information at the type level with runtime
    information?

    \medskip {\small e.g., productivity information, }

    \medskip {\small Conjecture: multi-modal type theory will work
      well for this (in place of $\mathcal{C}$)}
  \end{enumerate}
\end{frame}

\end{document}
