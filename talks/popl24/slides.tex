% -*- TeX-engine: xetex -*-

\documentclass[xetex,aspectratio=169,14pt,hyperref={pdfpagelabels=true,pdflang={en-GB}}]{beamer}

\definecolor{LinkBlue}{RGB}{53,72,207}
\definecolor{SlideBackground}{RGB}{251,247,240}
%\definecolor{GridColour}{RGB}{
\definecolor{TitleRed}{rgb}{0.6,0,0}
\definecolor{Constructor}{RGB}{221,34,221}


\newcommand{\tmRec}{\mathrm{rec}}
\newcommand{\tyNat}{\mathrm{Nat}}
\newcommand{\conZero}{\mathsf{zero}}
\newcommand{\conSucc}{\mathsf{succ}}
\newcommand{\conRefl}{\mathsf{refl}}

\usepackage[pattern=tri,fullpage,minorcolor=SlideBackground!50!black!15,majorcolor=black!10]{gridpapers}
\usepackage{euler}
\usepackage[cm-default]{fontspec}
\usepackage{xunicode}
\usepackage[normalem]{ulem}
\usepackage{mathpartir}

\usepackage{xltxtra}

\defaultfontfeatures{Ligatures=TeX}

\renewcommand{\arraystretch}{1.2}
\usepackage{booktabs}

\setmainfont{Linux Biolinum O}


\setbeamertemplate{navigation symbols}{}
\setbeamercolor{background canvas}{bg=SlideBackground}
\usecolortheme[rgb={0.6,0,0}]{structure}
\usefonttheme{serif}
\usefonttheme{structurebold}
\setbeamercolor{description item}{fg=black}

\newcommand{\mailto}[1]{\textcolor{LinkBlue}{\href{mailto:#1}{#1}}}

\newcommand{\deemph}[1]{\textcolor{black!60}{#1}}

\title{Polynomial Time and Dependent Types}
\author{{\bf Robert Atkey} \\
  {\it Mathematically Structured Programming group} \\
  {\it University of Strathclyde} \\
  \mailto{robert.atkey@strath.ac.uk}}
\date{Principles of Programming Languages (POPL)\\ \deemph{19th January 2024}}

\newenvironment{typings}%
{\begin{array}{@{}l@{\hspace{0.3em}}c@{\hspace{0.3em}}l}\ignorespaces}%
{\end{array}\ignorespacesafterend}

\newenvironment{lines}%
{\begin{array}{@{}l}\ignorespaces}%
{\end{array}\ignorespacesafterend}

\definecolor{kwcolour}{rgb}{0.5,0.5,0.8}

\newcommand{\Data}{\textcolor{kwcolour}{\bf\mathsf{data}}}
\newcommand{\Where}{\textcolor{kwcolour}{\bf\mathsf{where}}}
\newcommand{\Set}{\mathrm{Set}}
\newcommand{\Not}{\mathsf{not}}
\newcommand{\dname}[1]{\mathrm{#1}}
\newcommand{\String}{\dname{String}}
\newcommand{\cnstr}[1]{\textcolor{Constructor}{\mathsf{#1}}}
\newcommand{\Chu}{\mathrm{Chu}}
\newcommand{\defnd}{\stackrel{\mathit{def}}=}
\newcommand{\op}{\mathsf{op}}

\newcommand{\heading}[1]{\textcolor{TitleRed}{#1}}

\begin{document}

\begin{frame}
  \titlepage {\scriptsize Funded by EPSRC: EP/T026960/1 \emph{AISEC:
      AI Secure and Explainable by Construction}}
\end{frame}

\begin{frame}
  \textcolor{TitleRed}{Dependent Type Theory is}

  \bigskip

  A programming language

  \raggedleft \only<2->{So we can write programs}\only<1>{~}

  \raggedright

  \medskip

  {\it and} a proof language

  \raggedleft \only<3->{and reason about their extensional behaviour}\only<-2>{~}

%  \only<4->{but only the ``extensional behaviour''}\only<-3>{~}

  \pause \pause
\end{frame}

\newcommand{\comment}[1]{\textcolor{black!60}{#1}}
\newcommand{\hl}[1]{\textcolor{TitleRed}{\it #1}}

\begin{frame}
  What if we want to talk about programs' complexity?
\end{frame}

\begin{frame}

  Are costs \hl{Extrinsic} or \hl{Intrinsic}? \\
  \qquad \comment{do programs consume resources by a ``tick'' effect?} \\
  \qquad \comment{or simply by existing and executing?}

  \bigskip

  \hl{Explicit} or \hl{Implicit} accounting? \\
  \qquad \comment{``this program takes $n^2+3\frac{12}{27}n$ steps if the input is size $n$'', or} \\
  \qquad \comment{``this program is polytime for all inputs''}

  \bigskip

  How to mediate \hl{Intensional} and \hl{Extensional} meanings? \\
  \qquad \comment{How to stop costs interfering with extensional behaviour?}
\end{frame}

\begin{frame}
  \heading{This work:}

  \bigskip

  \begin{center}
    {\bf Implicit polynomial time} by linear typing

    $+$

    Extensional program reasoning with dependent types
  \end{center}

  \bigskip
  \pause

  \begin{center}
    Goal: {\it Synthetic} Computational Complexity Theory\pause(?)
  \end{center}
\end{frame}

\begin{frame}
  \begin{center}
    {\bf Implicit polynomial time} by linear typing

    \pause

    \comment{two different ways}
  \end{center}
\end{frame}

\begin{frame}
  \hl{\bf Polynomial time} is \emph{feasible} computation

  \bigskip
  \pause

  Feasible computation should be closed under:
  \begin{enumerate}
  \item<3-> Iteration over the input
  \item<4-> Composition
  \item<5-> Nesting of iterations
  \end{enumerate}

  \pause\pause\pause\pause
  \bigskip

  Let's build a type system for polynomial time using linear types.
\end{frame}

\newcommand{\TyNat}{\textsc{Nat}}

\begin{frame}
  Separate \hl{iterable} and \hl{non-iterable} types.

  \bigskip

  Iterable $\TyNat$:
\begin{displaymath}
  \inferrule*
  {\vdash M_z : A \\ x : A \vdash M_s : A \\ \Gamma \vdash N : \tyNat}
  {\Gamma \vdash \tmRec\,N\,\{\conZero \mapsto M_z; \conSucc(x) \mapsto M_s\} : A}
\end{displaymath}

  \bigskip

  Non-iterable $\TyNat^\circ$:
  \begin{displaymath}
    \inferrule*
    {\Gamma_1 \vdash M_z : A \\ \Gamma_2, x : \TyNat^\circ \vdash M_s : A \\ \Gamma_3 \vdash N : \TyNat^\circ}
    {\Gamma_1, \Gamma_2, \Gamma_3 \vdash \mathrm{case}\,N\,\{\conZero \mapsto M_z; \conSucc(x) \mapsto M_s\} : A}
  \end{displaymath}
\end{frame}

\begin{frame}
  \hl{NIMBYism}: {\bf Ban} construction \pause of $\TyNat$ values \\
  \qquad \comment{all $\TyNat$ must come from the ``outside''}

  \bigskip
  \pause

  Non-iterable $\TyNat^\circ$ may be constructed:
  \begin{mathpar}
    \mathit{zero} : \TyNat^\circ

    \mathit{succ} : \TyNat^\circ \multimap \TyNat^\circ
  \end{mathpar}

  Duplication of $\TyNat$ is permitted: $\mathit{dupNat} : \TyNat \multimap \TyNat \otimes \TyNat$ \\
  \qquad \comment{allows for nesting of iterations}

  \bigskip
  \pause

  Call this the \hl{Cons-free system}. \\
  \quad \hl{Sound} and \hl{Complete} for Polytime. \\
  \quad \comment{non-duplication of functions prevents encoding of constructors}
\end{frame}

\begin{frame}
  The \hl{Cons-free} system is not much fun to program in.

  \pause
  \bigskip

  \hl{LFPL} system: allow transmission of iterability via a $\Diamond$ type \\
  \qquad \comment{(Hofmann, 1999)} % FIXME: check

  \pause
  \bigskip

  Construction consumes $\Diamond$s:
  \begin{mathpar}
    \mathit{zero} : \Diamond \multimap \TyNat

    \mathit{succ} : \Diamond \multimap \TyNat \multimap \TyNat
  \end{mathpar}

  \pause

  Iteration yields $\Diamond$s:
  \begin{displaymath}
    \inferrule*
    {d : \Diamond \vdash M_z : A \\
      d : \Diamond, x : A \vdash M_s : A \\
      \Gamma \vdash N : \TyNat}
    {\Gamma \vdash \mathrm{iter}(d.~M_z, d~x.~M_s, N) : A}
  \end{displaymath}

  \pause

  \hl{Sound} and \hl{Complete} for Polytime. \comment{(retain non-iterable types.)}
\end{frame}

\begin{frame}
  \begin{center}
    Linear Dependent Types
  \end{center}
\end{frame}

\begin{frame}
  \emph{In} Type Theory

  \begin{displaymath}
    x_1 : S_1, \dots, x_n : S_n \vdash M : T
  \end{displaymath}

  \begin{center}
    variables $x_1, \dots, x_n$ are mixed usage

    some \emph{logical}, some \emph{computational}
  \end{center}

  \pause
  \bigskip

  \emph{In} Linear Logic

  \begin{displaymath}
    x_1 : X_1, \dots, x_n : X_n \vdash M : Y
  \end{displaymath}

  \begin{center}
    the presence of a variable $x$ records its usage

    each $x_i$ must be ``used'' by $M$ exactly once
  \end{center}

\end{frame}

% \begin{frame}
%   \begin{displaymath}
%     n : \mathit{Nat}, x : \mathsf{Fin}(n) \vdash x : \mathsf{Fin}(n)
%   \end{displaymath}

%   \pause
%   \bigskip

%   \hspace{15.5em}$x$ is used \hl{computationally}

%   \pause
%   \bigskip

%   \hspace{12em}$n$ is used \hl{logically}
% \end{frame}

\begin{frame}
\end{frame}

% \begin{frame}
%   \begin{displaymath}
%     n : \mathsf{Nat}, x : \mathsf{Fin}(n) \vdash x : \mathsf{Fin}(n)
%   \end{displaymath}

%   \bigskip

%   Can we read this judgement linearly?

%   \pause
%   \bigskip

%   $n$ appears in the context, but is not used computationally

%   \pause
%   \bigskip

%   $n$ appears \emph{twice} in types

%   \pause
%   \bigskip

%   Is $n$ even used at all?
% \end{frame}

% \begin{frame}
%   \begin{displaymath}
%     n : \mathsf{Nat} \mid x : \mathsf{Fin}(n) \vdash x : \mathsf{Fin}(n)
%   \end{displaymath}

%   \bigskip
%   \pause

%   Separate \emph{intuitionistic} / \emph{unrestricted} uses and \emph{linear} uses

%   \bigskip

%   Types can depend on intuitionistic data, but not linear data

%   % \raggedleft
%   % \textcolor{black!60}{\it will come back to this...}

%   % \raggedright

%   \bigskip

%   {\footnotesize
%     \textcolor{black!60}{(Barber, 1996)} \\
%     \textcolor{black!60}{(Cervesato and Pfenning, 2002)} \\
%     \textcolor{black!60}{(Krishnaswami, Pradic, and Benton, 2015)} \\
%     \textcolor{black!60}{(V{\'a}k{\'a}r, 2015)}\\
%   }
% \end{frame}


\newcommand{\point}{\textcolor{TitleRed}{$\rhd$}~}

\begin{frame}
  \begin{center}
    \hl{Quantitative Type Theory}

    \textcolor{black!60}{(McBride, 2016) (Atkey, 2018)}
  \end{center}
\end{frame}

% \begin{frame}
%   Quantitative Coeffect calculi:

%   \begin{displaymath}
%     x_1 \stackrel{\rho_1}: S_1, \dots, x_n \stackrel{\rho_n}: S_n \vdash M : T
%   \end{displaymath}

%   \bigskip

%   The $\rho_i$ record usage from some semiring $R$ \\
%   \quad . $1 \in R$ --- a use \\
%   \quad . $0 \in R$ --- not used \\
%   \quad . $\rho_1 + \rho_2$ --- adding up uses (e.g., in an application) \\
%   \quad . $\rho_1\rho_2$ --- nested uses

%   \bigskip

%   {\footnotesize
%     \textcolor{black!60}{(Petricek, Orchard, and Mycroft, 2014)} \\
%     \textcolor{black!60}{(Brunel, Gaboardi, Mazza, and Zdancewic, 2014)} \\
%     \textcolor{black!60}{(Ghica and Smith, 2014)} \\
%   }
% \end{frame}

\begin{frame}
  \hl{Quantitative} (non-dependent) calculi:
  \begin{displaymath}
    x_1 \stackrel{\rho_1}: S_1, \dots, x_n \stackrel{\rho_n}: S_n \vdash M : T
  \end{displaymath}
  The $\rho_i$ record usage from some semiring $R$ \\

  \bigskip
  \pause

  \hl{QTT}: allow $0$-usage data to appear in types.
  \textcolor{black!60}{(McBride, 2016)}

  \begin{displaymath}
    x_1 \stackrel{\rho_1}: S_1, \dots, x_n \stackrel{\rho_n}: S_n \vdash M \stackrel\sigma: T
  \end{displaymath}
  where $\sigma \in \{0,1\}$. \\
  \qquad \point $\sigma = 1$ --- the ``real'' computational world \\
  \qquad \point $\sigma = 0$ --- the types world \\
  {\footnotesize \textcolor{black!60}{(allowing arbitrary $\rho$ yields a system where substitution is inadmissible (Atkey, 2018))}}
\end{frame}

% \begin{frame}
%   \heading{Zero-ing is admissible}

%   \begin{displaymath}
%     \inferrule*
%     {\Gamma \vdash M \stackrel{1}: T}
%     {0\Gamma \vdash M \stackrel{0}: T}
%   \end{displaymath}

%   \bigskip

%   means that every linear term has an ``extensional'' counterpart (or constitutent)

%   \bigskip

%   which can be used at type checking time to construct types

%   \bigskip

%   has the effect of making the linear system a restriction of the intuitionistic
% \end{frame}

\begin{frame}
  \begin{center}
    QTT/LFPL
  \end{center}
\end{frame}

\newcommand{\istype}{\mathrm{type}}
\newcommand{\isctxt}{\mathrm{ctxt}}

\begin{frame}
  \begin{mathpar}
    \inferrule*
    {\Gamma~\isctxt}
    {0\Gamma \vdash \Diamond~\istype}

    \inferrule*
    {\Gamma~\isctxt}
    {0\Gamma \vdash * \stackrel0: \Diamond}

    \inferrule*
    {\Gamma \vdash M \stackrel0: \Diamond}
    {\Gamma \vdash M \equiv * \stackrel0: \Diamond}
  \end{mathpar}

  \bigskip

  In the $\sigma=0$ fragment, $\Diamond$s are free.
\end{frame}

\begin{frame}
  \begin{mathpar}
    \inferrule*
    {\Gamma \vdash M \stackrel\sigma: \Diamond}
    {\Gamma \vdash \conZero(M) \stackrel\sigma: \tyNat}

    \inferrule*
    {\Gamma_1 \vdash M \stackrel\sigma: \Diamond \\
      \Gamma_2 \vdash N \stackrel\sigma: \tyNat \\
      0\Gamma_1 = 0\Gamma_2}
    {\Gamma_1 + \Gamma_2 \vdash \conSucc(M,N) \stackrel\sigma: \tyNat}
  \end{mathpar}
\end{frame}

\newcommand{\redzero}{\textcolor{TitleRed}{\mathbf{0}}}
\newcommand{\redstar}{\colorbox{TitleRed}{\textcolor{white}{$*$}}}

\begin{frame}
  \begin{mathpar}
    \mprset{flushleft}
    \inferrule*
    {\redzero\Gamma, x \stackrel\redzero: \tyNat \vdash P~\istype \\\\
      \Gamma \vdash M \stackrel\sigma: \tyNat \\\\
      \redzero\Gamma, d \stackrel\sigma: \Diamond \vdash N_z \stackrel\sigma: P[\conZero(*)/x] \\\\
      \redzero\Gamma, d \stackrel\sigma: \Diamond, n \stackrel\redzero: \tyNat, p \stackrel\sigma: P[n/x] \vdash N_s \stackrel\sigma : P[\conSucc(*,n)/x]}
    {\Gamma \vdash \tmRec\,M\,\{\conZero(d) \mapsto N_z; \conSucc(d,n;p) \mapsto N_s\} \stackrel\sigma: P[M/x]}
  \end{mathpar}

  Remember: $\sigma \in \{0,1\}$.

  \bigskip

  When $\sigma = 0$, $\Diamond$s are free so this acts like the normal $\TyNat$-iterator.
\end{frame}

\begin{frame}
  \begin{mathpar}
    \mprset{flushleft}
    \inferrule*
    {\redzero\Gamma, x \stackrel\redzero: \tyNat \vdash P~\istype \\\\
      \Gamma \vdash M \stackrel\sigma: \tyNat \\\\
      \redzero\Gamma, d \stackrel\sigma: \Diamond \vdash N_z \stackrel\sigma: P[\conZero(\redstar)/x] \\\\
      \redzero\Gamma, d \stackrel\sigma: \Diamond, n \stackrel\redzero: \tyNat, p \stackrel\sigma: P[n/x] \vdash N_s \stackrel\sigma : P[\conSucc(\redstar,n)/x]}
    {\Gamma \vdash \tmRec\,M\,\{\conZero(d) \mapsto N_z; \conSucc(d,n;p) \mapsto N_s\} \stackrel\sigma: P[M/x]}
  \end{mathpar}

  Remember: $\sigma \in \{0,1\}$.

  \bigskip

  When $\sigma = 0$, $\Diamond$s are free so this acts like the normal $\TyNat$-iterator.
\end{frame}

\begin{frame}
  \hl{Soundness} (simplified)

  \begin{displaymath}
    0\Gamma, x \stackrel1: \tyNat \vdash M \stackrel1: \tyNat
  \end{displaymath}

  \bigskip

  Exists a realiser that always terminates in time polynomial in $x$.

  \quad \comment{Extension of Dal Lago and Hofmann (\comment{2011}) realisability}
\end{frame}

\newcommand{\Rtype}{\mathbf{R}}
\newcommand{\rIntro}{\mathbf{R}}
\newcommand{\rElim}{\mathbf{R}^{-1}}

\begin{frame}
  \textcolor{TitleRed}{Reflecting Realisability}

  \begin{mathpar}
    \inferrule*
    {0\Gamma \vdash A~\istype}
    {0\Gamma \vdash \Rtype(A)~\istype}

    \inferrule*
    {0\Gamma \vdash M \stackrel1: A}
    {0\Gamma \vdash \rIntro(M) \stackrel\sigma: \Rtype(A)}

    \inferrule*
    {\Gamma \vdash M \stackrel\sigma: \Rtype(A)}
    {\Gamma \vdash \rElim(M) \stackrel{\sigma'}: A}
  \end{mathpar}
\end{frame}

\begin{frame}
  \begin{center}
    Using QTT/LFPL for Complexity Theory
  \end{center}
\end{frame}

\newcommand{\BoolTy}{\mathrm{Bool}}
\newcommand{\cTrue}{\mathsf{true}}
\newcommand{\cFalse}{\mathsf{false}}
\newcommand{\ListTy}{\mathrm{List}}
\newcommand{\cNil}{\mathsf{nil}}
\newcommand{\cCons}{\mathsf{cons}}

\begin{frame}
  A \hl{problem} is a pair $(A : \mathsf{U}, P : A \to \mathsf{U})$.\\
  \qquad $A$ are the instances \\
  \qquad $P$ are the solutions

  \bigskip
  \pause

  $\begin{array}{@{}l}
      \mathrm{PTIME}(A,P) = \\
      \quad (f \stackrel1: \mathbf{R}(A \to \BoolTy)) \otimes \left((a \stackrel1: A) \to (\mathbf{R}^{-1}(f)\, a = \cTrue) \Leftrightarrow P\,a\right)
    \end{array}$

  \bigskip
  \pause

  $\begin{array}{@{}l}
    (A,P) \stackrel{\textsc{Poly}}\Rightarrow (B,Q) = \\
    \quad (f \stackrel1: \mathbf{R}(A \to B)) \otimes \left((a \stackrel1: A) \to Q(\mathbf{R}^{-1}(f)\,a) \Leftrightarrow P\,a\right)
  \end{array}$
\end{frame}

\begin{frame}
  $\begin{array}{l}
    \textbf{data}\,\mathrm{ND}\,(A : \mathsf{U}) : \mathsf{U}\,\textbf{where} \\
    \quad \mathsf{return} : A \to \mathrm{ND}\,A \\
    \quad \mathsf{choice} : (\BoolTy \to \mathrm{ND}\,A) \to \mathrm{ND}\,A
  \end{array}$

  \pause
  \bigskip

  $\mathrm{runWithOracle} \stackrel0: \mathrm{ND}\,A \to \ListTy(\BoolTy) \to \mathrm{Maybe}\,A$

  \pause
  \bigskip

  $\begin{array}[t]{@{}l}
      \mathrm{NP}(A,P) = \\
      (f \stackrel1: \mathbf{R}(A \to \mathrm{ND}(\BoolTy)))\, \otimes \\
      \, (a \stackrel1: A) \to \\
      \, \left((\mathit{bs} \stackrel1: \ListTy(\BoolTy)) \otimes (\mathrm{runWithOracle}\,(\mathbf{R}^{-1}(f)\,a)\,\mathit{bs} = \mathsf{just}\,\cTrue)\right) \Leftrightarrow P\,a
    \end{array}$
\end{frame}

\begin{frame}
  $\begin{array}{@{}l}
    \textbf{data}\,\mathrm{Dist}\,(A : \mathsf{U}) : \mathsf{U}\,\textbf{where} \\
    \quad \mathsf{return} : A \to \mathrm{Dist}\,A \\
    \quad \mathsf{choice} : \mathbb{Q}[0,1] \to (\BoolTy \to \mathrm{Dist}\,A) \to \mathrm{Dist}\,A
  \end{array}$

  \bigskip
  \pause

  $\mathrm{probTrue} \stackrel0: \mathrm{Dist}\,\BoolTy \to \mathbb{Q}[0,1]$

  \bigskip
  \pause

  $\begin{array}[t]{@{}l}
      \mathrm{BPP}(A,P) = \\
      \quad (f \stackrel1: \mathbf{R}(A \to \mathrm{Dist}(\BoolTy))) \, \otimes \\
      \qquad (a \stackrel1: A) \to
      (\mathrm{probTrue}\,(\mathbf{R}^{-1}(f)\,a) \geq \frac{2}{3}) \Leftrightarrow P\,a
    \end{array}$

\end{frame}

\begin{frame}
  % Summary
\end{frame}

\begin{frame}
  \textcolor{TitleRed}{What's in the paper?}

  \begin{enumerate}
  \item \hl{QTT/Cons-free} and \hl{QTT/LFPL} \comment{full rules, more datatypes}

  \item Programming with QTT/LFPL.

  \item Synthetic definitions of \hl{P}, \hl{NP}, \hl{BPP}, using effects.

  \item Soundness proof with Agda formalisation \\
    \comment{Realisability proof extending Dal Lago and Hofmann (2011)}

  \end{enumerate}

  \bigskip

  \textcolor{TitleRed}{What next?}
  \begin{enumerate}
  \item More complexity classes (\textsc{coNP}, \textsc{PSpace}, ...)
  \item Polytime circuit generators?
  \item What is needed for a proof of NP-completeness of 3-SAT?
  \item Application to formalised cryptography?
  \item Explicit resource accounting (counting diamonds)
  \end{enumerate}
\end{frame}

\end{document}
